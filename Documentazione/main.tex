\documentclass{article}

\usepackage{style,xcolor}

% Sub-preambles
% https://github.com/MartinScharrer/standalone

% Encodings
\usepackage{amsmath,amssymb,gensymb,textcomp}

% Better tables
% Wide tables go to https://tex.stackexchange.com/q/332902
\usepackage{array,multicol,multirow,siunitx,tabularx}
\usepackage{tabularray}

% Better enum
\usepackage{enumitem}

% Graphics
\usepackage{caption,float}

%lastpage
\usepackage{lastpage}

% Allow setting >max< width of figure
% Remove if unneeded
\usepackage[export]{adjustbox} % 'export' allows adjustbox keys in \includegraphics
\usepackage{mwe}

% Custom commands
% Remove if unneeded
\newcommand*\mean[1]{\bar{#1}}

\ocorso
\oreporttype{}
\title{ \emph{Ratatouille23}}
\reportlayout%

\numberwithin{equation}{section}
\numberwithin{figure}{section}

\begin{document}

\coverpage%

\section*{Gruppo INGSW2223\_N\_04}
\begin{center}
  \begin{tblr}{hlines = {0.9pt}, vlines = {0.9pt}, row{1} = {marroneApp!60}, colspec = {X[c]X[c]X[c]}, width = \textwidth}
        \textbf{No.} & \textbf{Nome e Cognome} & \textbf{Matricola} \\
        1            & Luigi Vessella          & N86003354\\
        2            & Matteo Marino           & N86003963\\
        3            & Biagio Speranza         & N86002964\\
  \end{tblr}
\end{center}

\newpage
\tableofcontents
\newpage
\section{Descrizione del progetto}
    {\large 
        Ratatouille23 è un sistema finalizzato al supporto alla gestione e all’operatività di attività di
        ristorazione. Il sistema consiste in un’applicazione performante e affidabile, attraverso cui gli utenti
        possono fruire delle funzionalità del sistema in modo intuitivo, rapido e piacevole. 
        La nostra visione della richiesta prevede lo sviluppo di un'applicazione mobile su sistema operativo Android che offrirà agli utilizzatori i seguenti servizi: 
    }
    
    \begin{itemize}
        \item  \emph{ I proprietari (o amministratori) potranno creare account per i dipendenti}
        \item  \emph{I proprietari saranno in grado di gestire uno o più ristoranti}
        \item  \emph{Si avrà la possibilità di visualizzare e/o modificare il menu}
        \item  \emph{I dipendenti (camerieri) saranno in grado di prendere e inoltrare le ordinazioni in cucina}
        \item  \emph{Gli addetti alla cucina potranno avvisare i camerieri nel momento in cui è pronta un'ordinazione}
        \item  \emph{Tutti potranno visionare lo storico delle ordinazioni con i dettagli}
    \end{itemize} 
    
    \begin{flushleft}
      {\large Ovviamente, il tipo di funzionalità messo a disposizione dall'applicativo sarà cambiato
        dinamicamente a seconda di chi si logga. 
        Gli amministratori e/o supervisori saranno inoltre dotati di tablet con a bordo l'OS di Google Android per
        una migliore fruizione della loro dashboard.}

      \vspace{1cm}

    {\large I dipendenti quali camerieri, operatori di cucina, capisala, saranno dotati di smartphone aziendali sempre con OS Android correttamente configurati per un'ottimale fruizione dell'applicazione.}
      \vspace{1cm}

    {\large \textbf{Tutti i dispositivi dovranno essere in grado di accedere a internet, preferibilmente per tutta la durata del servizio. Il funzionamento del server è invece garantito servizi allo stato dell'arte quali Microsoft Azure.}}
    \end{flushleft}
    
    \subsection{Premessa e presentazione}
    \begin{figure}[H]
        \centering
        \includegraphics[scale=0.5]{assets/immagini varie/D.Norman grafico.png}
        \caption{\textbf{Grafico}: Evoluzione dei prodotti high-tech secondo D.Norman}\label{fig:Mockup_Homepage}
    \end{figure}

    \begin{flushleft}
        
        Con questo piccolo preambolo vogliamo richiamare l'attenzione sul diagramma di evoluzione dei prodotti software di D.Norman, il quale mostra il ciclo di vita che solitamente questo tipo di prodotto ha.
        All'inizio avremo un'app più incentrata sulle tecnologie implementate e/o da implementare per  "cercare " di raggiungere ciò che l'utente medio richiede dall'applicativo. Successivamente si otterrà il punto di pareggio in cui si soddisfano
        i bisogni dell'utente tipico, e ciò potrebbe essere considerato un buon punto di equilibrio del sistema.
        In fine, sempre ammesso che l'app sia sopravvissuta al mercato durante questo ciclo, ci può essere una fase finale con una condizione
        di iperfunzionalità, dove si cercherà via via di soddisfarre le esigenze di una platea di utenti più ampia. Nel caso della nostra applicazione Ratatouille23 descritta in questo documento, pensiamo di distribuire un prodotto nella sua 
        fase di equilibrio (vedere grafico) che certamente soddisfa le esigenze degli attori utilizzatori descritti nei punti successivi, ma che allo stesso tempo può essere facilmente aggiornata e migliorata qualora dovessero aggiungersi nuovi target d'utenti o cambiare
        gli esistenti.
    \end{flushleft}
 
    \section{Modello Funzionale}
    \subsection{Requisiti Funzionali}
        \begin{flushleft}  
            {\large
                Vengono qui presentati i requisiti funzionali dell'applicativo, ossia quei servizi che l'app  deve offrire agli utenti:
            } 
        \end{flushleft}
        
        \subsubsection{Admin}
        

        \begin{center}
          % Admin 1
          \begin{tblr}{hlines = {0.9pt}, vlines = {0.9pt}, row{1} = {marroneApp!60}, colspec = {X[c]X[l]}, width = \textwidth}
                  \textbf{ID}         & Admin\_1                             \\
                  \textbf{Nome}       & Registrazione account amministratore \\
                  \textbf{Descrizione} & {Il sistema permette ad un amministratore non registrato di registrarsi alla\\ piattaforma utilizzando: \textit{nome}, \textit{cognome}, \textit{email}, \textit{codice fiscale}, \textit{P.IVA} e \textit{password}}
          \end{tblr}

          \vspace{1cm}

          % Admin 2
          \begin{tblr}{hlines = {0.9pt}, vlines = {0.9pt}, row{1} = {marroneApp!60}, colspec = {X[c]X[l]}, width = \textwidth}
                  \textbf{ID}         & Admin\_2                             \\
                  \textbf{Nome}       & Modifica account amministratore \\
                  \textbf{Descrizione} & {Il sistema permette ad un amministratore loggato di modifcare i campi del proprio account}
          \end{tblr}

          \vspace{1cm}

          % Admin 3
          \begin{tblr}{hlines = {0.9pt}, vlines = {0.9pt}, row{1} = {marroneApp!60}, colspec = {X[c]X[l]}, width = \textwidth}
                  \textbf{ID}         & Admin\_3                             \\
                  \textbf{Nome}       & Registrazione dei dipendenti \\
                  \textbf{Descrizione} & {Il sistema permette ad un amministratore di creare utenze per i dipendenti non registrati del ristorante,\\ specificandone \textit{nome}, \textit{cognome}, \textit{email} e \textit{ruolo}}
          \end{tblr}

          \vspace{1cm}

          % Admin 4
          \begin{tblr}{hlines = {0.9pt}, vlines = {0.9pt}, row{1} = {marroneApp!60}, colspec = {X[c]X[l]}, width = \textwidth}
                  \textbf{ID}          & Admin\_4                             \\
                  \textbf{Nome}        & Modificare/Eliminare account dipendenti  \\
                  \textbf{Descrizione} & {Il sistema permette ad un amministratore loggato di modificare gli account dei propri dipendenti.}
          \end{tblr}

          \vspace{1cm}

          % Admin 5
          \begin{tblr}{hlines = {0.9pt}, vlines = {0.9pt}, row{1} = {marroneApp!60}, colspec = {X[c]X[l]}, width = \textwidth}
                  \textbf{ID}          & Admin\_5                             \\
                  \textbf{Nome}        & Aggiungere personale della cucina  \\
                  \textbf{Descrizione} & {Il sistema permette ad un amministratore loggato di aggiungere al proprio ristorante il personale della cucina.}
          \end{tblr}

          \vspace{1cm}

          % Admin 6
          \begin{tblr}{hlines = {0.9pt}, vlines = {0.9pt}, row{1} = {marroneApp!60}, colspec = {X[c]X[l]}, width = \textwidth}
                  \textbf{ID}          & Admin\_6                             \\
                  \textbf{Nome}        & Aggiunta dei Ristoranti  \\
                  \textbf{Descrizione} & {Il sistema permette ad un amministratore di poter aggiungere le proprie attività di ristorazione (CAMPI DA DEFINIRE).}
          \end{tblr}

          \vspace{1cm}

          % Admin 7
          \begin{tblr}{hlines = {0.9pt}, vlines = {0.9pt}, row{1} = {marroneApp!60}, colspec = {X[c]X[l]}, width = \textwidth}
                  \textbf{ID}          & Admin\_7                             \\
                  \textbf{Nome}        &  Modifica/Eliminazione dei Ristoranti  \\
                  \textbf{Descrizione} & {Il sistema permette ad un amministratore di poter modificare ed eliminare le proprie attività di ristorazione del sistema.}
          \end{tblr}

          \vspace{1cm}

          % Admin 8
          \begin{tblr}{hlines = {0.9pt}, vlines = {0.9pt}, row{1} = {marroneApp!60}, colspec = {X[c]X[l]}, width = \textwidth}
                  \textbf{ID}          & Admin\_8                             \\
                  \textbf{Nome}        &  Modifica dati dei Dipendenti  \\
                  \textbf{Descrizione} & {Il sistema permette ad un amministratore loggato di poter cambiare i dati personali dei dipendenti (nome, cognome, email, luogo).}
          \end{tblr}

          \vspace{1cm}

          % Admin 9
          \begin{tblr}{hlines = {0.9pt}, vlines = {0.9pt}, row{1} = {marroneApp!60}, colspec = {X[c]X[l]}, width = \textwidth}
                  \textbf{ID}          & Admin\_9                             \\
                  \textbf{Nome}        &  Aggiungere/Modificare elementi nel menù  \\
                  \textbf{Descrizione} & {Il sistema permette ad un amministratore o supervisore dell'attività di ristorazione di aggiungere/modificare elementi nel menù dell'attività. Ogni elemento dovrà avere i seguenti campi:\\ Nome, Costo, Descrizione, Elenco di Allergeni, Categoria/e}
          \end{tblr}

          \vspace{1cm}

          % Admin 10
          \begin{tblr}{hlines = {0.9pt}, vlines = {0.9pt}, row{1} = {marroneApp!60}, colspec = {X[c]X[l]}, width = \textwidth}
                  \textbf{ID}          & Admin\_10                             \\
                  \textbf{Nome}        &  Modifica dati personali\\
                  \textbf{Descrizione} & {Il sistema permette ad un amministratore loggato di poter cambiare i propri dati  personali.}
          \end{tblr}

          \vspace{1cm}

          % Admin 12
          \begin{tblr}{hlines = {0.9pt}, vlines = {0.9pt}, row{1} = {marroneApp!60}, colspec = {X[c]X[l]}, width = \textwidth}
                  \textbf{ID}          & Admin\_12                             \\
                  \textbf{Nome}        &  Tradurre il menu           \\
                  \textbf{Descrizione} & {Il sistema permette ad un Amministratore di poter tradurre gli elementi del proprio menù in un altra lingua}
          \end{tblr}

          \vspace{1cm}

          % Admin 13
          \begin{tblr}{hlines = {0.9pt}, vlines = {0.9pt}, row{1} = {marroneApp!60}, colspec = {X[c]X[l]}, width = \textwidth}
                  \textbf{ID}          & Admin\_13                             \\
                  \textbf{Nome}        &  Visualizza statistiche personale della cucina\\
                  \textbf{Descrizione} & {Il sistema permette ad un Amministratore di visualizzare, grazie anche all'ausilio di grafici interattivi, informazioni sull'operato degli addetti alla cucina}
          \end{tblr}

          \vspace{1cm}

          % Admin 14
          \begin{tblr}{hlines = {0.9pt}, vlines = {0.9pt}, row{1} = {marroneApp!60}, colspec = {X[c]X[l]}, width = \textwidth}
                  \textbf{ID}          & Admin\_14                             \\
                  \textbf{Nome}        & Modifica menu \\
                  \textbf{Descrizione} & {Il sistema permette ad un Amministratore di poter aggiornare (aggiungere, modificare ed eliminare elementi) il menu del ristorante}
          \end{tblr}
        \end{center}

        \subsubsection{Cameriere}
        \begin{center}

          % Waiter 1
          \begin{tblr}{hlines = {0.9pt}, vlines = {0.9pt}, row{1} = {marroneApp!60}, colspec = {X[c]X[l]}, width = \textwidth}
                  \textbf{ID}          & Waiter\_1                             \\
                  \textbf{Nome}        & Prendere le ordinazioni \\
                  \textbf{Descrizione} & {Il sistema permette ai camerieri di prendere ordinazioni ai tavoli, inoltrandole alla cucina.}
          \end{tblr}

          \vspace{1cm}

          % Waiter 2
          \begin{tblr}{hlines = {0.9pt}, vlines = {0.9pt}, row{1} = {marroneApp!60}, colspec = {X[c]X[l]}, width = \textwidth}
                  \textbf{ID}          & Waiter\_2                             \\
                  \textbf{Nome}        & Gestione delle ordinazioni \\
                  \textbf{Descrizione} & {Il sistema permette ai camerieri di prendere ordinazioni ai tavoli, inoltrandole alla cucina.}
          \end{tblr}

          \vspace{1cm}

          % Waiter 3
          \begin{tblr}{hlines = {0.9pt}, vlines = {0.9pt}, row{1} = {marroneApp!60}, colspec = {X[c]X[l]}, width = \textwidth}
                  \textbf{ID}          & Waiter\_3                             \\
                  \textbf{Nome}        & Evasione degli ordini \\
                  \textbf{Descrizione} & {Il sistema permette a un Cameriere di marcare i singoli elementi di un ordine come conclusi, aggiornando gli addetti in cucina}
          \end{tblr}

          \vspace{1cm}

          % Waiter 4
          \begin{tblr}{hlines = {0.9pt}, vlines = {0.9pt}, row{1} = {pink!60}, colspec = {X[c]X[l]}, width = \textwidth}
                  \textbf{ID}          & Waiter\_4                             \\
                  \textbf{Nome}        & Sollecitare la cucina \\
                  \textbf{Descrizione} & {Il sistema permette a un Cameriere di sollecitare la cucina nel caso in cui un ordine sia da troppo tempo in preparazione }
          \end{tblr}

        \end{center}
        
        \subsubsection{Cucina}

        \begin{center}
          % Kitchen 1
          \begin{tblr}{hlines = {0.9pt}, vlines = {0.9pt}, row{1} = {marroneApp!60}, colspec = {X[c]X[l]}, width = \textwidth}
                  \textbf{ID}          & Kitchen\_1                             \\
                  \textbf{Nome}        & Marcare gli ordini pronti \\
                  \textbf{Descrizione} & {Il sistema permette alla cucina di marcare gli ordini pronti alla "consegna", specificando lo chef che l'ha preparato}
          \end{tblr}

          \vspace{1cm}

          % Kitchen 2
          \begin{tblr}{hlines = {0.9pt}, vlines = {0.9pt}, row{1} = {marroneApp!60}, colspec = {X[c]X[l]}, width = \textwidth}
                  \textbf{ID}          & Kitchen\_2                             \\
                  \textbf{Nome}        & Sollecitare i camerieri \\
                  \textbf{Descrizione} & {Il sistema permette alla cucina di notificare i camerieri nel caso in cui un ordine sia pronto alla consegna da troppo tempo }
          \end{tblr}

        \end{center}

        \subsubsection{Supervisore}

        \begin{center}

          % Hypervisor 1 (Forse Supervisor (?))
          \begin{tblr}{hlines = {0.9pt}, vlines = {0.9pt}, row{1} = {marroneApp!60}, colspec = {X[c]X[l]}, width = \textwidth}
                  \textbf{ID}          & Hypervisor\_1                             \\
                  \textbf{Nome}        & Avvisare il personale \\
                  \textbf{Descrizione} & {Il sistema permette ad un supervisore ed un amministratore di inviare degli avvisi al personale}
          \end{tblr}

          \vspace{1cm}

          % Hypervisor 2 (Forse Supervisor (?))
          \begin{tblr}{hlines = {0.9pt}, vlines = {0.9pt}, row{1} = {marroneApp!60}, colspec = {X[c]X[l]}, width = \textwidth}
                  \textbf{ID}          & Hypervisor\_2                             \\
                  \textbf{Nome}        & Visualizzare stato ordini per tavolo\\
                  \textbf{Descrizione} & {Il sistema permette ad un Supervisore ed un Cameriere di visualizzare gli stati delle ordinazioni per ogni tavolo}
          \end{tblr}

          \vspace{1cm}

          % Hypervisor 3 (Forse Supervisor (?))
          \begin{tblr}{hlines = {0.9pt}, vlines = {0.9pt}, row{1} = {marroneApp!60}, colspec = {X[c]X[l]}, width = \textwidth}
                  \textbf{ID}          & Hypervisor\_3                             \\
                  \textbf{Nome}        & Visualizzare ordini in arrivo\\
                  \textbf{Descrizione} & {Il sistema permette ad un Supervisore e alla Cucina di visualizzare l'elenco di ordini in arrivo}
          \end{tblr}

          \vspace{1cm}

          % Hypervisor 4 (Forse Supervisor (?))
          \begin{tblr}{hlines = {0.9pt}, vlines = {0.9pt}, row{1} = {marroneApp!60}, colspec = {X[c]X[l]}, width = \textwidth}
                  \textbf{ID}          & Hypervisor\_4                             \\
                  \textbf{Nome}        & Visualizzare ordini in uscita\\
                  \textbf{Descrizione} & {Il sistema permette ad un Supervisore e alla Cucina di visualizzare l'elenco di ordini pronti all'uscita}
          \end{tblr}

          \vspace{1cm}

          % Hypervisor 5 (Forse Supervisor (?))
          \begin{tblr}{hlines = {0.9pt}, vlines = {0.9pt}, row{1} = {marroneApp!60}, colspec = {X[c]X[l]}, width = \textwidth}
                  \textbf{ID}          & Hypervisor\_5                             \\
                  \textbf{Nome}        & Visualizzare storico ordini\\
                  \textbf{Descrizione} & {Il sistema permette ad un Supervisore e alla Cucina di visualizzare lo storico degli ordini}
          \end{tblr}

          \vspace{1cm}
        \end{center}


        \subsubsection{Tutti}

        \begin{center}

          % All 1
          \begin{tblr}{hlines = {0.9pt}, vlines = {0.9pt}, row{1} = {marroneApp!60}, colspec = {X[c]X[l]}, width = \textwidth}
                  \textbf{ID}          & All\_1                             \\
                  \textbf{Nome}        & Recupero/Cambio Password\\
                  \textbf{Descrizione} & {Il sistema permette a tutti gli utenti registrati di poter recuperare la password e di poterla modificare}
          \end{tblr}

          \vspace{1cm}

          % All 2
          \begin{tblr}{hlines = {0.9pt}, vlines = {0.9pt}, row{1} = {marroneApp!60}, colspec = {X[c]X[l]}, width = \textwidth}
                  \textbf{ID}          & All\_2                             \\
                  \textbf{Nome}        & Login\\
                  \textbf{Descrizione} & {Il sistema permette a tutti gli utenti registrati di poter effettuare il login con Username e Password all'interno della piattaforma}
          \end{tblr}

          \vspace{1cm}

          % All 3
          \begin{tblr}{hlines = {0.9pt}, vlines = {0.9pt}, row{1} = {marroneApp!60}, colspec = {X[c]X[l]}, width = \textwidth}
                  \textbf{ID}          & All\_3                             \\
                  \textbf{Nome}        & Visualizzare un avviso o sollecitazione\\
                  \textbf{Descrizione} & {Il sistema permette a tutti gli utenti registrati pi poter visualizzare gli avvisi e le sollecitazioni ricevuti, marcandoli come visualizzati}
          \end{tblr}
        \end{center}

        \newpage

        \subsection{Requisiti Non Funzionali}
        \begin{flushleft} Vengono qui elencati i requisiti non funzionali dell'applicativo: \end{flushleft}

        \begin{center}
          % Unfunctional 1
          \begin{tblr}{hlines = {0.9pt}, vlines = {0.9pt}, row{1} = {marroneApp!60}, colspec = {X[c]X[l]}, width = \textwidth}
                  \textbf{ID}          & Unfunctional\_1                             \\
                  \textbf{Nome}        & Policy first password\\
                  \textbf{Descrizione} & {Il sistema richiede al primo accesso di un dipendente il cambio della password provvisoria in una password personale.}
          \end{tblr}

          \vspace{1cm}

          % Unfunctional 2
          \begin{tblr}{hlines = {0.9pt}, vlines = {0.9pt}, row{1} = {marroneApp!60}, colspec = {X[c]X[l]}, width = \textwidth}
                  \textbf{ID}          & Unfunctional\_2                             \\
                  \textbf{Nome}        & Verifica esistenza della mail\\
                  \textbf{Descrizione} & {Al fine di evitare registrazioni con e-mail fittizie, il sistema richiede l'autenticazione della mail mediante codice di verifica per poter procedere con il completamento della registrazione}
          \end{tblr}

          \vspace{1cm}

          % Unfunctional 3
          \begin{tblr}{hlines = {0.9pt}, vlines = {0.9pt}, row{1} = {marroneApp!60}, colspec = {X[c]X[l]}, width = \textwidth}
                  \textbf{ID}          & Unfunctional\_3                             \\
                  \textbf{Nome}        & Password Strength\\
                  \textbf{Descrizione} & {Al fine di evitare la creazione di password poco sicure, il sistema impone all'utente di utilizzare una password di almeno 8 caratteri che contenga numeri e caratteri speciali.}
          \end{tblr}

          \vspace{1cm}

          % Unfunctional 4
          \begin{tblr}{hlines = {0.9pt}, vlines = {0.9pt}, row{1} = {marroneApp!60}, colspec = {X[c]X[l]}, width = \textwidth}
                  \textbf{ID}          & Unfunctional\_4                             \\
                  \textbf{Nome}        & \textcolor{red}{TODO}\\
                  \textbf{Descrizione} & {Una P.IVA appartiene ad un solo amministratore.}
          \end{tblr}
        \end{center}

        \subsection{Requisiti di Dominio}

        \begin{center}
          % Domain 1
          \begin{tblr}{hlines = {0.9pt}, vlines = {0.9pt}, row{1} = {marroneApp!60}, colspec = {X[c]X[l]}, width = \textwidth}
                  \textbf{ID}          & Domain\_1                             \\
                  \textbf{Nome}        & GDPR\\
                  \textbf{Descrizione} & {Il sistema deve essere conforme alla normativa GDPR (Regolamento Generale  sulla Protezione dei Dati), per il trattamento dei dati personali e riguardante la privacy dell'utente}
          \end{tblr}

          \vspace{1cm}

          % Domain 2
          \begin{tblr}{hlines = {0.9pt}, vlines = {0.9pt}, row{1} = {marroneApp!60}, colspec = {X[c]X[l]}, width = \textwidth}
                  \textbf{ID}          & Domain\_2                             \\
                  \textbf{Nome}        & NORME ISO\\
                  \textbf{Descrizione} & {\textcolor{red}{TODO}}
          \end{tblr}
        \end{center}

    \subsection{Modellazione dei casi d'uso}

    \begin{flushleft}
        In questa sezione vengono riportati i Use-case diagram relativi al sistema e le tabelle di Cockburn e i Sequence diagram relativi alle funzionalità \textit{\textbf{Aggiungi ristorante}}, \textit{\textbf{Aggiungi piatto}}, \textit{\textbf{Prendi ordinazione}}, \textit{\textbf{Visualizza avvisi}}.
    \end{flushleft}

    \subsubsection{Use-Case Diagram}
        \begin{flushleft}
            Qui viene riportato lo Use Case Diagram relativo all'applicazione, che per rendere la lettura più semplificata 
            è stato diviso in più sezioni.
        \end{flushleft}
        
        \begin{figure}[H]
            \centering
            \includegraphics[width=0.8\textwidth]{assets/diagrammi/Use-Case/Use-Case Generale.png}
            \caption{Use Case dell'applicazione}
            \label{fig:ucdGenerale}
        \end{figure}
        
        \begin{figure}[H]
            \centering
            \includegraphics[width=0.8\textwidth]{assets/diagrammi/Use-Case/Modifica Profilo.png}
            \caption{Use Case relativo alla modifica del profilo}
            \label{fig:ucdModProfile}
        \end{figure}

        \begin{figure}[H]
            \centering
            \includegraphics[width=0.8\textwidth]{assets/diagrammi/Use-Case/Gestione ristoranti.png}
            \caption{Use Case relativo alla gestione dei ristoranti}
            \label{fig:ucdResturantMgmt}
        \end{figure}

        \begin{figure}[H]
            \centering
            \includegraphics[width=0.7\textwidth]{assets/diagrammi/Use-Case/Gestione ordini.png}
            \caption{Use Case relativo alla gestione degli ordini}
            \label{fig:ucdOrderMgmt}
        \end{figure}
        
        \begin{figure}[H]
            \centering
            \includegraphics[width=0.6\textwidth]{assets/diagrammi/Use-Case/Gestione dipendenti.png}
            \caption{Use Case relativo alla gestione dei dipendenti}
            \label{fig:ucdWorkersMgmt}
        \end{figure}

        \begin{figure}[H]
            \centering
            \includegraphics[width=0.8\textwidth]{assets/diagrammi/Use-Case/Gestione avvisi.png}
            \caption{Use Case relativo alla gestione degli avvisi}
            \label{fig:ucdAdvMgmt}
        \end{figure}
\newpage
    \include{capitoli/Documento dei requisiti software/TabelleCockburn}
    \subsection{Mock-up dell'applicazione}
    \begin{flushleft}
        Qui vengono presentati i mockup relativi all'applicazione.
    \end{flushleft}
    \subsubsection{Homepage dell'applicazione}
        \begin{figure}[H]
            \centering
            \includegraphics[width=0.70\textwidth]{assets/Mockup/Mockup_Homepage.png}
            \caption{\textbf{M01}: Homepage dell'applicazione}
            \label{fig:Mockup_Homepage}
        \end{figure}
        \begin{flushleft}
            \textbf{ID} \ \Large{\textit{\textbf{M01}}}\\
        \end{flushleft}
        \textbf{Componenti}:\\
        \begin{tabular}{lll}
            \hline
            \textbf{Tipo}   &   \textbf{Nome}   &   \textbf{Funzione} \\
            \hline
            Bottone       &   ACCEDI &   Quando cliccato porta alla schermata \textit{\textbf{M02}} \\
            \hline
            Bottone & REGISTRATI   &   Quando cliccato porta alla schermata \textit{\textbf{M03}} \\
            \hline
        \end{tabular}
        \newpage
        \subsubsection{Schermata di accesso nel sistema}
            \begin{figure}[H]
                \centering
                \includegraphics[width=0.70\textwidth]{assets/Mockup/Mockup_Accesso.png}
                \caption{\textbf{M02}: Schermata di accesso nel sistema}
                \label{fig:Mockup_Login}
            \end{figure}
            \begin{flushleft}
                \textbf{ID} \ \Large{\textit{\textbf{M02}}}\\
            \end{flushleft}
            \textbf{Componenti}:\\
            \begin{tabular}{lll}
                \hline
                \textbf{Tipo}   &   \textbf{Nome}   &   \textbf{Funzione} \\
                \hline
                Edit Text       &   EMAIL &   Permette l'inserimento dell'email dell'utente \\
                \hline
                Edit Text & PASSWORD  &  Permette l'inserimento della password dell'utente  \\
                \hline
                Bottone &   ENTRA   & Quando cliccato porta alla schermata \textit{\textbf{M04}} se admin, \textit{\textbf{M0?}} se dipendente \\
                \hline
            \end{tabular}
        \newpage
        \subsubsection{Schermata di registrazione nel sistema}
        \begin{figure}[H]
            \centering
            \includegraphics[width=0.60\textwidth]{assets/Mockup/Mockup_Registrazione.png}
            \caption{\textbf{M03}: Schermata di registrazione nel sistema}
            \label{fig:Mockup_Register}
        \end{figure}
        \begin{flushleft}
            \textbf{ID} \ \Large{\textit{\textbf{M03}}}\\
            \large{\textit{Nota}: la schermata di registrazione è valida solo per gli admin proprietari dei ristoranti, in quanto sono poi loro a registrare i dipendenti.}\\
        \end{flushleft}
        \textbf{Componenti}:\\
        \begin{tabular}{lll}
            \hline
            \textbf{Tipo}   &   \textbf{Nome}   &   \textbf{Funzione} \\
            \hline
            Edit Text    &   NOME    &   Permette di inserire il nome dell'admin \\
            \hline
            Edit Text & COGNOME   &  Permette di inserire il cognome dell'admin \\
            \hline
            Edit Text    &   PASSWORD    &   Permette di inserire una password per l'admin \\
            \hline
            Edit Text    &   EMAIL   &   Permette di inserire l'email dell'admin \\
            \hline
            Edit Text    & CODICE FISCALE    & Permette di inserire il codice fiscale dell'admin \\
            \hline
            Edit Text    &   P.IVA   & Permette di inserire la partita IVA dell'admin \\
            \hline
            Bottone &   REGISTRATI  & Quando cliccato riporta alla schermata \textit{\textbf{M01}} per permettere l'accesso \\
            \hline
        \end{tabular}
        \newpage
        \subsubsection{Schermata home per gli admin}
        \begin{figure}[H]
            \centering
            \includegraphics[width=0.70\textwidth]{assets/Mockup/Mockup_AdminDashboard.png}
            \caption{\textbf{M04}: Schermata home per gli admin}
            \label{fig:Mockup_AdminDashboard}
        \end{figure}
        \begin{flushleft}
            \textbf{ID} \ \Large{\textit{\textbf{M04}}} \\
        \end{flushleft}
        \textbf{Componenti}:\\
        \begin{tabular}{lll}
            \hline
            \textbf{Tipo}   &   \textbf{Nome}   &   \textbf{Funzione} \\
            \hline
            \multirow{2}*{ScrollView}&   \multirow{2}*{I TUOI RISTORANTI}    &   Visualizza ed eventualmente permette la modifica dei\ \ \ \ \ \ \  \\ && ristoranti registrati    \\
            \hline
            Bottone    &   AGGIUNGI RISTORANTE    &   Quando cliccato porta alla schermata \textit{\textbf{M05}} \\
            \hline
            Bottone    &   MODIFICA   &   Quando cliccato porta alla schermata \textit{\textbf{M06}} \\
            \hline
        \end{tabular}
        \newpage
        \subsubsection{Schermata di registrazione di un nuovo ristorante}
        \begin{figure}[H]
            \centering
            \includegraphics[width=0.65\textwidth]{assets/Mockup/Mockup_AddResturant.png}
            \caption{\textbf{M05}: Schermata di registrazione di un nuovo ristorante}
            \label{fig:Mockup_AddResturant}
        \end{figure}
        \begin{flushleft}
            \textbf{ID} \ \Large{\textit{\textbf{M05}}}
        \end{flushleft}
        \textbf{Componenti}:\\
        \begin{tabular}{lll}
            \hline
            \textbf{Tipo}   &   \textbf{Nome}   &   \textbf{Funzione} \\
            \hline
            Edit Text   &   NOME    &   Permette di inserire il nome del nuovo ristorante\\
            \hline
            Edit Text   &   LOCALITA'   &   Permette di inserire la località del nuovo ristorante\\
            \hline
            Edit Text   &   COPERTI    & Permette di inserire il n° dei coperti del nuovo ristorante\\
            \hline
            \multirow{2}*{Bottone} &   \multirow{2}*{SALVA}    &   Quando cliccato salva il nuovo ristorante nel database e torna alla schermata \\ && \textit{\textbf{M04}} \\
            \hline
        \end{tabular}
        \newpage
        \subsubsection{Schermata di modifica di un ristorante}
        \begin{figure}[H]
            \centering
            \includegraphics[width=0.70\textwidth]{assets/Mockup/Mockup_ResturantManager.png}
            \caption{\textbf{M06}: Schermata di modifica di un ristorante}
            \label{fig:Mockup_ResturantManager}
        \end{figure}
        \begin{flushleft}
            \textbf{ID} \ \Large{\textit{\textbf{M06}}}
        \end{flushleft}
        \textbf{Componenti}:\\
        \begin{tabular}{lll}
            \hline
            \textbf{Tipo}   &   \textbf{Nome}   &   \textbf{Funzione} \\
            \hline
            Bottone   &   AGGIUNGI CAMERIERE &   Quando cliccato porta alla schermata \textit{\textbf{M07}}\\
            \hline
            Bottone   &   AGGIUNGI MENU &   Quando cliccato porta alla schermata \textit{\textbf{M08}}\\
            \hline
        \end{tabular}
        \newpage
        \subsubsection{Schermata di registrazione di un nuovo dipendente}
        \begin{figure}[H]
            \centering
            \includegraphics[width=0.60\textwidth]{assets/Mockup/Mockup_SaveWaiter.png}
            \caption{\textbf{M07}: Schermata di registrazione di un nuovo dipendente}
            \label{fig:Mockup_SaveWorker}
        \end{figure}
        \begin{flushleft}
            \textbf{ID} \ \Large{\textit{\textbf{M07}}}
        \end{flushleft}
        \textbf{Componenti}:\\
        \begin{tabular}{lll}
            \hline
            \textbf{Tipo}   &   \textbf{Nome}   &   \textbf{Funzione} \\
            \hline
            Edit Text   &   NOME    &   Permette di inserire il nome del nuovo dipendente\\
            \hline
            Edit Text   &   COGNOME   &   Permette di inserire il cognome del nuovo dipendente\\
            \hline
            \multirow{2}*{Edit Text}   &   \multirow{2}*{PASSWORD}    &   Permette di inserire la password temporanea del nuovo \\ && dipendente  \\
            \hline
            Edit Text   &   EMAIL   & Permette di inserire l'email del nuovo dipendente\\
            \hline
            Edit Text   &   CODICE FISCALE    &   Permette di inserire il codice fiscale del nuovo dipendente \\
            \hline
            \multirow{2}*{Spinner} &   \multirow{2}*{Waiter}    &   \multirow{2}*{Permette di inserire il ruolo del nuovo dipendente} \\ & (default) & \\
            \hline
            \multirow{2}*{Bottone} &   \multirow{2}*{REGISTRA}    &   Quando cliccato, se tutti i dati sono corretti, riporta alla \\ && schermata \textit{\textbf{M04}} registrando il nuovo dipendente \\
            \hline
        \end{tabular}
        \newpage
        \subsubsection{Schermata di gestione del menù}
        \begin{figure}[H]
            \centering
            \includegraphics[width=0.70\textwidth]{assets/Mockup/Mockup_MenuManager.png}
            \caption{\textbf{M08}: Schermata di gestione del menù}
            \label{fig:Mockup_MenuManager}
        \end{figure}
        \begin{flushleft}
            \textbf{ID} \ \Large{\textit{\textbf{M08}}}
        \end{flushleft}
        \textbf{Componenti}:\\
        \begin{tabular}{lll}
            \hline
            \textbf{Tipo}   &   \textbf{Nome}   &   \textbf{Funzione} \\
            \hline
            Bottone   &   AGGIUNGI PRODOTTO    &   Quando cliccato porta alla schermata \textit{\textbf{M0?}}\\
            \hline
            \multirow{2}*{Edit Text}   &   \multirow{2}*{GENERA MENU}   &   Permette di generare il menù e salvarlo sul dispositivo in \\ && formato PDF\\
            \hline
        \end{tabular}
    \subsection{Valutazione dell'usabilità}

    \begin{flushleft}
       Per la valutazione dell'usabilità del nostro applicativo a priori, cioè prima della fase di sviluppo vera a propria,
       abbiamo deciso di imporci come linee guida le euristiche di Nielsen.
       Ne sono 10, ma vorremmo richiamare l'attenzione su alcune di esse nello specifico:
        \begin{itemize}
            \item \textit{Visibilità dello stato del sistema.} Il sistema presentava una discreta mancanza di feedback, che prevediamo di colmare con elementi quali Dialog, Toast, AlertDialog e SnackBar.
            \item \textit{Prevenzione degli errori.} Il sistema reagisce in maniera controllata e predeterminata alle situazioni di errori che gli utenti possono causare. Nulla è lasciato al caso, ed è, nelle build provate dal team, gestita qualsiasi azione eseguibile dagli utenti.
            \item \textit{Riconoscere piuttosto che ricordare.} La nostra app è dotata di sezioni ben distinte, interfacce dinamiche a seconda del tipo di utente che le usa, e icone e testi ecplicativi dell'azione che si va a intrapendere.
            \item \textit{Guida e documentazione.} E' presente un simpatico topolino (che richiama il logo dell'app) che consiglia tramite vignette e piccoli dialoghi le azioni che si possono intraprendere. Inoltre ci sono piccole note come campi obbligatori ecc.
        \end{itemize}
    \end{flushleft}

    \begin{figure}[H]
        \centering
        \includegraphics[scale=0.6]{assets/immagini varie/grafico usabilita.png}
        \caption{\textbf{Grafico}: Usabilità}\label{fig:Usabilità_graph}
    \end{figure}
    \subsection{Individuazione target d'utenti}
    \begin{flushleft}
        La conoscenza dell'utente finale è di importanza fondamentale per chi progetti sistemi software di questo tipo. La grande diversità degli esseri umani
        fa sì che, anche considerando compiti e contesti d'uso simili, un oggetto potrebbe risultare usabile per un certo utente e
        del tutto inusabile per un altro.\\
        Sicuramente, in base alle alle richieste del committente abbiamo subito individuato 4 principali categorie di utenti utilizzatori dell' app:\vspace{0.5cm}
       
        \textbf{Admin:} Amministratore e proprietario del ristorante. Una persona che deve avere tutto sotto controllo, può gestire le sue attività nonchè i 
        dipendenti che ne fanno parte, aggiungerne di nuova e talvolta, purtroppo, eliminarli.\vspace{1cm}

        \textbf{Supervisore:} Dopo l'amministratore, è nella "gerarchia" da noi definita, la seconda persona con più funzioni disponibili in-app. 
        Anch'esso disponde di una dashboard completa sullo stille dell'admin.\vspace{1cm}

        \textbf{Cameriere/Addetto sala:} Senza la figura del cameriere un'attività di ristorazione non va avanti. Sappiamo quant'è importante fornire a questi ultimi
        un applicativo funzionale, facile da usare e da apprendere: per questo la sua interfaccia è ottimizzata per un palmare o smartphone compatto.\vspace{1cm}


        \textbf{Addetto Cucina:} Riceve tutti gli ordini dai camerieri e li inoltra alla cucina. Anche lui dispone di un'interfaccia semplice e dinamica che perfettamente si adatta al suo ruolo nell'attività.
        \vspace{0.5cm}

        
        
        Tutto ciò è reso possibile da uno sviluppo che va incontro alle esigenze dei diversi utenti. La nostra interfaccia riconosce il tipo di utente che logga e cambia
        in base alle sue esigenze. 

    \end{flushleft}


    \begin{figure}[H]
        \centering
        \includegraphics[scale=0.5]{assets/immagini varie/target utenti.png}
        \caption*{\textbf{Figura}: Da una visione centrata sul sistema a una visione centrata sull’utente}\label{fig:target_utenti}
    \end{figure}
        
    %TODO
    \subsection{Glossario}
    \begin{flushleft}
        In questa sezione vengono chiariti alcuni termini usati all'interno della documentazione, per rendere la lettura accessibile
        anche ai non esperti del settore.
    \end{flushleft}

    \begin{itemize}
        \item \emph{RecyclerView:} potente approccio nella risoluzione di un problema comune: la creazione di liste per la visualizzazione (su Android) di dati ottenuti da un servizio remoto o da un database locale.
        \item \emph{Adapter:} Un oggetto di tipo adapter in Android rappresenta un ponte tra un' AdapterView e i dati che questa deve rappresentare.
        \item \emph{Dialog:} Tipo di popup che il sistema Android mette a disposizione che mostra una finestra di dialog personalizzabile.Nella nostra app, mostra "Ok" e "Torna indietro".
        \item \emph{Spring Boot:} Spring è uno strumento che permette a noi programmatori di usare il linguaggio di programmazione ad oggetti java per scrivere ottime app lato server.
        \item \emph{JPA:} Le Java Persistence API, talvolta riferite come JPA, sono un framework per il linguaggio di programmazione Java che si occupa della gestione della persistenza dei dati di un DBMS relazionale
        \item \emph{MVC:} Pattern Model-View-Controller.
        \item \emph{Three-Tier Architecture:} l'espressione architettura three-tier ("a tre strati") indica una particolare architettura software e hardware di tipo multi-tier per l'esecuzione di un'applicazione web che prevede la suddivisione dell'applicazione in tre diversi 
        moduli o strati dedicati rispettivamente alla interfaccia utente, alla logica funzionale (business logic) e alla gestione dei dati persistenti. Android e Spring-Boot ne supportano i principi.
    \end{itemize}
    \subsection{Class diagram di analisi o dominio}

    \begin{flushleft}
        Class diagram di analisi servono a bla bla bla\ldots
        Some description over here\ldots
    \end{flushleft}

    \subsubsection{Class diagram Login}
        \begin{figure}[H]
            \centering
            \includegraphics[scale=0.5]{assets/diagrammi/Class diagram di analisi/Gestione Login.png}
            \caption{\textbf{C01}: Class diagram Login}\label{fig:Login}
        \end{figure}

    \subsubsection{Class diagram Ristorante}
        \begin{figure}[H]
            \centering
            \includegraphics[scale=0.5]{assets/diagrammi/Class diagram di analisi/Gestione ristorante.png}
            \caption{\textbf{C02}: Class diagram Gestione ristorante}\label{fig:Ristorante}
        \end{figure}

    \subsubsection{Class diagram dipendenti}
        \begin{figure}[H]
            \centering
            \includegraphics[scale=0.5]{assets/diagrammi/Class diagram di analisi/Gestione dipendenti.png}
            \caption{\textbf{C03}: Class diagram gestione dipendenti}\label{fig:Dipendenti}
        \end{figure}
    
    \subsubsection{Class diagram Menu}
        \begin{figure}[H]
            \centering
            \includegraphics[scale=0.5]{assets/diagrammi/Class diagram di analisi/Gestione menu.png}
            \caption{\textbf{C04}: Class diagram gestione menu}\label{fig:Menu}
        \end{figure} 

    \subsubsection{Class diagram Avvisi}
        \begin{figure}[H]
            \centering
            \includegraphics[scale=0.5]{assets/diagrammi/Class diagram di analisi/Gestione Avvisi.png}
            \caption{\textbf{C05}: Class diagram gestione avvisi}\label{fig:Avvisi}
        \end{figure}

    \subsubsection{Class diagram Ordini}
        \begin{figure}[H]
            \centering
            \includegraphics[scale=0.5]{assets/diagrammi/Class diagram di analisi/Gestione ordini.png}
            \caption{\textbf{C06}: Class diagram gestione ordini}\label{fig:Ordini}
        \end{figure}

    \subsubsection{Class diagram statistiche}
        \begin{figure}[H]
            \centering
            \includegraphics[scale=0.5]{assets/diagrammi/Class diagram di analisi/Gestione Stat.png}
            \caption{\textbf{C07}: Class diaram gestione statistiche}\label{fig:Statistiche}
        \end{figure}
    \subsection{Sequence di analisi}

    \begin{flushleft}
        Abbiamo scelto di modellare (in fase di analisi dei requisiti) con i sequence diagram i segenti casi d'uso:
        \begin{itemize}
            \item  \emph{Aggiunta menu del ristorante}
            \item  \emph{Prendere oridinazione}
        \end{itemize}
    \end{flushleft}

    \begin{figure}[H]
        \centering
        \includegraphics[scale=0.6]{assets/diagrammi/Sequence di analisi/sequence_add_menu.png}
        \caption{\textbf{Sequence}: Aggiungi menu}\label{fig:seq_add_menu}
    \end{figure}
   
    
\end{document}