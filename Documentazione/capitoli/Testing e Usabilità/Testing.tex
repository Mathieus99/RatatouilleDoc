\section{Testing xUnit}

\lstset{
tabsize = 4, %% set tab space width
showstringspaces = false, %% prevent space marking in strings, string is defined as the text that is generally printed directly to the console
numbers = left, %% display line numbers on the left
commentstyle = \color{green}, %% set comment color
keywordstyle = \color{blue}, %% set keyword color
stringstyle = \color{red}, %% set string color
rulecolor = \color{black}, %% set frame color to avoid being affected by text color
basicstyle = \small \ttfamily, %% set listing font and size
breaklines = true, %% enable line breaking
numberstyle = \tiny,
}
 

\begin{flushleft}
    In questa sezione presentiamo quella che forse è una delle parti fondamentali nel ciclo di sviluppo di un software: il testing.
    Ne esistono di vari tipi, per "semplicità" qui si richiede il testing di unità, effettuato con jUnit.
\end{flushleft}


%PRIMO METODO

\subsection{Test funzione check credenziali}
\begin{flushleft}
    \textbf{Abbiamo utilizzato il modo} "Modo"\\
    \textbf{Classi di equivalenza individuate:} "classi"
\end{flushleft}
\vspace{0.2cm}

\begin{lstlisting}[language = Java , frame = trBL , firstnumber = last , escapeinside={(*@}{@*)}]

    public class CheckCredentialsTest {

    //  CODICI DI ERRORE: 9 = PASSWORD VUOTA, 10 = PASSWORD TROPPO CORTA, 11 = PASSWORD NON VALIDA (REGEX)
    //                    12 = EMAIL VUOTA, 13 = EMAIL NON VALIDA

    public ArrayList<Integer> codici_errore = new ArrayList<Integer>();

    @AfterEach
    public void clearArrayList() {
        codici_errore.clear();
    }

    @Test
    public void testCheckCredentials(){
        assertEquals(codici_errore, checkCredentials("ser.dimartino@studenti.unina.it", "Password.123"));
    }

    @Test
    public void tesPasswordVuota(){
        codici_errore.add(9);
        assertAll(
                () -> assertEquals(codici_errore, checkCredentials("fra.cutugno@studenti.unina.it", null)),
                () -> assertEquals(codici_errore, checkCredentials("lu.starace@studenti.unina.it", ""))
        );
    }

    @Test
    public void testPasswordTroppoCorta(){
        codici_errore.add(10);
        assertEquals(codici_errore, checkCredentials("gio.cutolo@studenti.unina.it", "Aa_09."));
    }

    @Test
    public void testPasswordNonValida(){
        codici_errore.add(11);
        assertAll(
                () -> assertEquals(codici_errore, checkCredentials("alb.aloisio@studenti.unina.it", "password?123")),  // MANCA LA MAIUSCOLA
                () -> assertEquals(codici_errore, checkCredentials("an.corazza@studenti.unina.it", "PASSWORD#123")),  //MANCA LA MINUSCOLA
                () -> assertEquals(codici_errore, checkCredentials("alb.aloisio@studenti.unina.it", "Password123")),  // MANCA IL CARATTERE SPECIALE
                () -> assertEquals(codici_errore, checkCredentials("an.corazza@studenti.unina.it", "Password_unoduetre")) // MANCA IL NUMERO
        );

    }

    @Test
    public void testEmailVuota(){
        codici_errore.add(12);
        assertAll(
                () -> assertEquals(codici_errore, checkCredentials(null, "Password.123")),
                () -> assertEquals(codici_errore, checkCredentials("", "Password.123"))
        );
    }

    @Test
    public void testEmaildNonValida(){
        codici_errore.add(13);
        assertAll(
                () -> assertEquals(codici_errore, checkCredentials("emailcompletamentesbagliata", "Password.123")),  // EMAIL COMPLETAMENTE SBAGLIATA
                () -> assertEquals(codici_errore, checkCredentials("@gmail.com", "Password.123")),  // MANCA LO USERNAME
                () -> assertEquals(codici_errore, checkCredentials("biagio@.net", "Password.123")),  // MANCA IL SECOND LEVEL DOMAIN
                () -> assertEquals(codici_errore, checkCredentials("matteo@libero.", "Password.123")), // MANCA IL TOP LEVEL DOMAIN
                () -> assertEquals(codici_errore, checkCredentials("luigivirgilio.it", "Password.123")), //MANCA LA @
                () -> assertEquals(codici_errore, checkCredentials("MATTEO[BIAGIO]LUIGI_@libero.IT", "Password.123")) // LO USERNAME PRESENTA CARATTERI NON CORRETTI
        );
    }

    @Test
    public void testErroriMultipli_9_12(){
        codici_errore.add(9);
        codici_errore.add(12);

        assertEquals(codici_errore, checkCredentials("", ""));
    }

    @Test
    public void testErroriMultipli_9_13(){
        codici_errore.add(9);
        codici_errore.add(13);

        assertEquals(codici_errore, checkCredentials("@studenti.@libero@com", ""));
    }

    @Test
    public void testErroriMultipli_10_12(){
        codici_errore.add(10);
        codici_errore.add(12);

        assertEquals(codici_errore, checkCredentials("", "Ab.34"));
    }

    @Test
    public void testErroriMultipli_10_13(){
        codici_errore.add(10);
        codici_errore.add(13);

        assertEquals(codici_errore, checkCredentials("emailcompletamentesbagliata", "Ab.34"));
    }

    @Test
    public void testErroriMultipli_11_12(){
        codici_errore.add(11);
        codici_errore.add(12);

        assertEquals(codici_errore, checkCredentials("", "Password.Password"));
    }

    @Test
    public void testErroriMultipli_11_13(){
        codici_errore.add(11);
        codici_errore.add(13);

        assertEquals(codici_errore, checkCredentials("@studenti.@libero@com", "@studenti.it"));
    }
}

\end{lstlisting}



%SECONDO METODO
\subsection{Test funzione che calcola incasso in un range di date.}
\begin{flushleft}
    \textbf{Abbiamo utilizzato il modo} "Modo"\\
    \textbf{Classi di equivalenza individuate:} "classi"
\end{flushleft}
\vspace{0.2cm}

\begin{lstlisting}[language = Java , frame = trBL , firstnumber = last , escapeinside={(*@}{@*)}]
public class getIncassoRangeGiorniTest {

    StatisticsActivityMock statisticsActivityMock;
    ArrayList<OrdineMock> ordiniM;
    DateTimeFormatter formatter;

    @Before
    public void setUp(){
        statisticsActivityMock = new StatisticsActivityMock();
        ordiniM = new ArrayList<>();
        formatter = DateTimeFormatter.ofPattern("yyyy-MM-dd");
    }

    @AfterEach
    public void clearArrayList(){
        ordiniM.clear();
    }


    @Test
    public void testGetIncassoRangeGiorni() {
        ordiniM.add(new OrdineMock(3, "2023-05-04"));
        ordiniM.add(new OrdineMock(105, "2023-05-04"));
        ordiniM.add(new OrdineMock(72, "2023-02-04"));

        LocalDate dataInizio = LocalDate.parse("2023-02-05", formatter);

        int result = statisticsActivityMock.getIncassoRangeGiorni(dataInizio, ordiniM);
        assertEquals(108, result);
    }

    @Test
    public void testZeroOrdini()  {
        LocalDate dataInizio = LocalDate.parse("2023-02-01", formatter);

        int result = statisticsActivityMock.getIncassoRangeGiorni(dataInizio, ordiniM);
        assertEquals(0, result);

    }

    @Test
    public void testOrdiniNull()  {
        LocalDate dataInizio = LocalDate.parse("2023-02-01", formatter);

        int result = statisticsActivityMock.getIncassoRangeGiorni(dataInizio, null);
        assertEquals(0, result);

    }

    @Test
    public void testDataInizioNull()  {
        ordiniM.add(new OrdineMock(3, "2023-05-04"));
        ordiniM.add(new OrdineMock(105, "2023-05-04"));
        ordiniM.add(new OrdineMock(72, "2023-02-04"));

        assertThrows(NullPointerException.class,
                () -> statisticsActivityMock.getIncassoRangeGiorni(null, ordiniM));
    }

    @Test
    public void testDataInizioFutura() {
        ordiniM.add(new OrdineMock(3, "2023-05-04"));
        ordiniM.add(new OrdineMock(105, "2023-05-04"));
        ordiniM.add(new OrdineMock(72, "2023-02-04"));

        LocalDate dataInizio = LocalDate.parse("2033-02-05", formatter);

        int result = statisticsActivityMock.getIncassoRangeGiorni(dataInizio, ordiniM);
        assertEquals(0, result);
    }

    @Test
    public void testDataInizioIrrealistica() {
        ordiniM.add(new OrdineMock(3, "2023-05-04"));
        ordiniM.add(new OrdineMock(105, "2023-05-04"));
        ordiniM.add(new OrdineMock(72, "2023-02-04"));

        LocalDate dataInizio = LocalDate.parse("0133-02-05", formatter);

        int result = statisticsActivityMock.getIncassoRangeGiorni(dataInizio, ordiniM);
        assertEquals(180, result);
    }

    @Test
    public void testDataOrdiniInFormatoSbagliato(){
        ArrayList<OrdineMock> ordiniM_1 = new ArrayList<>();
        ArrayList<OrdineMock> ordiniM_2 = new ArrayList<>();
        ArrayList<OrdineMock> ordiniM_3 = new ArrayList<>();
        ArrayList<OrdineMock> ordiniM_4 = new ArrayList<>();

        ordiniM_1.add(new OrdineMock(3, "2023"));
        ordiniM_2.add(new OrdineMock(105, "02/05/2023"));
        ordiniM_3.add(new OrdineMock(72, "due-aprile-2023"));
        ordiniM_4.add(new OrdineMock(105, "2023-32-31"));


        LocalDate dataInizio = LocalDate.parse("2023-02-01", formatter);

        assertAll(
                () ->   assertThrows(DateTimeParseException.class,
                        () -> statisticsActivityMock.getIncassoRangeGiorni(dataInizio, ordiniM_1)
                ),
                () ->   assertThrows(DateTimeParseException.class,
                        () ->statisticsActivityMock.getIncassoRangeGiorni(dataInizio, ordiniM_2)
                ),
                () ->   assertThrows(DateTimeParseException.class,
                        () -> statisticsActivityMock.getIncassoRangeGiorni(dataInizio, ordiniM_3)
                ),
                () ->   assertThrows(DateTimeException.class,
                        () -> statisticsActivityMock.getIncassoRangeGiorni(dataInizio, ordiniM_4)
                )

        );
        ordiniM_1.clear();
        ordiniM_2.clear();
        ordiniM_3.clear();
        ordiniM_4.clear();
    }

}

\end{lstlisting}


%TERZO METODO

\subsection{Test funzione che controlla i campi di un ristorante.}
\begin{flushleft}
    \textbf{Abbiamo utilizzato il modo}Valutando l'implementazione del metodo verso cui stiamo eseguendo del testing con strategia Black-Box, riteniamo che il WEAK EQUIVALENCE CLASS TESTING sia il criterio di copertura più indicato, fornendo
    8 Test Cases che riteniamo siano sufficienti a ritenere la batteria di testing esaustiva\\

    \textbf{Classi di equivalenza individuate:}\\ getRestaurantFieldsError è un metodo che prende 4 parametri di tipo Stringa: Nome, Coperti, Indirizzo, Numero di telefono: N, C, I, T.
    Il valore di ritorno del metodo sarà un ArrayList di interi dove ognuno conterrà un valore.

    Per ogni campo possimo identificare delle classi di equivalenza:
    \begin{itemize}
        \item N = \{VALIDO, TROPPO CORTO, NULL\} - TROPPO CORTO: Nome compostoda un solo carattere
        \item C = \{VALIDO, FUORI RANGE, NON VALIDO, NULL\} - FUORI RANGE: C > 1000 || C < 5 - NON VALIDO: C contiene o è interamente composto da caratteri non numerici
        \item I = \{VALIDO, TROPPO CORTO, NON VALIDO NULL\}- TROPPO CORTO: I < 5 - NON VALIDO: I contiene o è composto da caratteri speciali
        \item T = \{VALIDO, NON VALIDO, NULL\} - NON VALIDO: T non rispetta la regex che controlla che sia effettivamente un numero di telefono valido\\

    \end{itemize}
    

\end{flushleft}
\vspace{0.2cm}

\begin{lstlisting}[language = Java , frame = trBL , firstnumber = last , escapeinside={(*@}{@*)}]
public class getRestaurantFiedlsErrorsTest {

    public ArrayList<Integer> codici_errore = new ArrayList<Integer>();


    // L'ARRAYLIST DEVE ESSERE PULITO OGNI VOLTA CHE VIENE CONCLUSO UN CASO DI TEST
    @AfterEach
    public void clearArrayList(){
        codici_errore.clear();
    }

    @Test
    public void testgetRestaurantFieldsError(){
        ArrayList<Integer> actualErrors = getRestaurantFieldsErrors("Ristorante Test", "10", "Via Roma 1", "0123456789");
        assertEquals(codici_errore, actualErrors); //Funziona poiche non ci sono codici di errore: l'ArrayList risulta vuoto
    }

    @Test
    public void testNomeCampoTroppoCorto() {
        codici_errore.add(1);
        ArrayList<Integer> actualErrors = getRestaurantFieldsErrors("a", "10", "Via Roma 1", "0123456789");
        assertEquals(codici_errore, actualErrors);
    }

    @Test
    public void testCampoCopertiFuoriRange(){
        codici_errore.add(4);
        assertAll(
                () -> assertEquals(codici_errore, getRestaurantFieldsErrors("Ristorante Test", "1", "Via Roma 1", "0123456789")),
                () -> assertEquals(codici_errore, getRestaurantFieldsErrors("Ristorante Test", "100000", "Via Roma 1", "0123456789"))
        );
    }

    @Test
    public void testCampoCopertiNonValido(){
        codici_errore.add(9);
        assertAll(
                () -> assertEquals(codici_errore, getRestaurantFieldsErrors("Ristorante Test", "dieci", "Via Roma 1", "0123456789")),
                () -> assertEquals(codici_errore, getRestaurantFieldsErrors("Ristorante Test", "10a", "Via Roma 1", "0123456789"))
        );
    }


    @Test
    public void testLocazioneCampoTroppoCorto() {
        codici_errore.add(6);
        ArrayList<Integer> actualErrors = getRestaurantFieldsErrors("Ristorante Test", "10", "Via", "0123456789");
        assertEquals(codici_errore, actualErrors);
    }

    @Test
    public void testLocazioneNonValido() {
        codici_errore.add(10);
        ArrayList<Integer> actualErrors = getRestaurantFieldsErrors("Ristorante Test", "10", "Via_Napoli?", "0123456789");
        assertEquals(codici_errore, actualErrors);
    }


    @Test
    public void testNumeroTelefonoCampoTroppoCorto() {
        codici_errore.add(8);
        ArrayList<Integer> actualErrors = getRestaurantFieldsErrors("Ristorante Test", "10", "Via Roma 1", "12345678");
        assertEquals(codici_errore, actualErrors);
    }

    @Test
    public void testTuttiICampiVuoti() {
        codici_errore.add(2);
        codici_errore.add(3);
        codici_errore.add(5);
        codici_errore.add(7);
        ArrayList<Integer> actualErrors = getRestaurantFieldsErrors("", "", "", "");
        assertEquals(codici_errore, actualErrors);
    }
}
\end{lstlisting}



%QUARTO METODO


\subsection{Testing delle funzione media (in statistiche).}
\begin{flushleft}
    \textbf{Abbiamo utilizzato il modo} "Modo"\\
    \textbf{Classi di equivalenza individuate:} "classi"
\end{flushleft}
\vspace{0.2cm}
\begin{lstlisting}[language = Java , frame = trBL , firstnumber = last , escapeinside={(*@}{@*)}]
public class mediaTest {

    StatisticsActivityMock mock;

    @Before
    public void setUp(){
        mock = new StatisticsActivityMock();
    }

    //TESTING BLACK BOX

    @Test
    public void testMedia(){
        float result = mock.media(17, 50f);
        assertEquals(2.94f, result, 0.001f);
    }
    @Test
    public void testGiornoNegativoIncassoPositivo(){
        assertThrows(IllegalArgumentException.class,
                () -> mock.media(-3, 50.06f)
        );
    }
    @Test
    public void testGiornoPositivoIncassoNegativo(){
        assertThrows(IllegalArgumentException.class,
                () -> mock.media(3, -50.06f)
        );
    }
    @Test
    public void testGiornoEIncassoNegativi(){
        assertThrows(IllegalArgumentException.class,
                () -> mock.media(-3, -50.06f)
        );
    }

    @Test
    public void testGiornoZero(){
        assertAll(
                () ->  assertThrows(ArithmeticException.class,
                        () -> mock.media(0, 50.06f)
                ),
                () ->  assertThrows(ArithmeticException.class,
                        () -> mock.media(0, -50.06f)
                )
        );
    }

    //TESTING WHITE BOX

    @Test
    public void testGiornoNull(){
        Integer giorno = null;
        assertAll(
                () ->   assertThrows(NullPointerException.class,
                        () -> mock.media(giorno, 0.30f)
                ),
                () ->   assertThrows(NullPointerException.class,
                        () -> mock.media(giorno, -0.30f)
                )
        );
    }
    @Test
    public void testIncassoNull(){
        Float incasso = null;
        assertAll(
                () ->   assertThrows(NullPointerException.class,
                        () -> mock.media(0, incasso)
                ),
                () ->   assertThrows(NullPointerException.class,
                        () -> mock.media(3, incasso)
                ),
                () ->   assertThrows(NullPointerException.class,
                        () -> mock.media(-3, incasso)
                )
        );
    }

    @Test
    public void testCampiNull(){
        Integer giorno = null;
        Float incasso = null;
        assertThrows(NullPointerException.class,
                () -> mock.media(giorno, incasso)
        );
    }


}

\end{lstlisting}


