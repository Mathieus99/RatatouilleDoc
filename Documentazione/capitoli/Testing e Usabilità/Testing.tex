\subsection{Testing xUnit}

\lstset{
tabsize = 4, %% set tab space width
showstringspaces = false, %% prevent space marking in strings, string is defined as the text that is generally printed directly to the console
numbers = left, %% display line numbers on the left
commentstyle = \color{green}, %% set comment color
keywordstyle = \color{blue}, %% set keyword color
stringstyle = \color{red}, %% set string color
rulecolor = \color{black}, %% set frame color to avoid being affected by text color
basicstyle = \small \ttfamily, %% set listing font and size
breaklines = true, %% enable line breaking
numberstyle = \tiny,
}

\begin{flushleft}
    In questa sezione presentiamo quella che forse è una delle parti fondamentali nel ciclo di sviluppo di un software: il testing.
    Ne esistono di vari tipi, per "semplicità" qui si richiede il testing di unità, effettuato con jUnit.
\end{flushleft}

\subsubsection{Test funzione che controlla il codice fiscale.}

\begin{lstlisting}[language = Java , frame = trBL , firstnumber = last , escapeinside={(*@}{@*)}]
    public class CodiceFiscaleTest {
    @Test
    public void testCodiceFiscaleValido() {
        String cf = "SPRBGI99P20F839N"; //BIAGIO SPERANZA 20/09/1999 - NAPOLI
        assertTrue(isCodiceFiscaleValido(cf));
    }

    @Test
    public void testCodiceFiscaleValidoConSpazio() {
        String cf = "SPRBGI9 9P20F8 39N"; //CODICE FISCALE CORRETTO MA CON AGGIUNTA DI SPAZI
        assertFalse(isCodiceFiscaleValido(cf));
    }

    @Test
    public void testCodiceFiscaleNonValido() {
        String cf = "BPRBGI99P20F839N"; //VARIAZIONE DI UN CARATTERE DEL CODICE FISCALE CORRETTO
        assertFalse(isCodiceFiscaleValido(cf));
    }

    @Test
    public void testCodiceFiscaleInventato() {
        String cf = "BBHFTI09V87H011N"; //CODICE FISCALE COMPLETAMENTE INVENTATO
        assertFalse(isCodiceFiscaleValido(cf));
    }

    @Test
    public void testCodiceFiscaleCaratteriNonValidi() {
        String cf = "$PRBGI.9P20F839%"; //CODICE FISCALE CON CARATTERI NON VALIDI
        assertFalse(isCodiceFiscaleValido(cf));
    }

    @Test
    public void testCodiceFiscaleConCaratteriEscape() {
        String cf = "SPR\u0009BGI99P20F839N"; //CODICE FISCALE CON UN CARATTERE DI ESCAPE
        assertFalse(isCodiceFiscaleValido(cf));
    }

    @Test
    public void testCodiceFiscaleLunghezzaNonCorretta() {
        assertAll(
                () -> assertFalse(isCodiceFiscaleValido("SPRBGI99P20F839N4VB")), // CODICE FISCALE TROPPO LUNGO
                () -> assertFalse(isCodiceFiscaleValido("SP")) // CODICE FISCALE TROPPO CORTO
        );
    }

    @Test
    public void testCodiceFiscaleNull() {
        String cf = null; //CODICE FISCALE NULL
        assertFalse(isCodiceFiscaleValido(cf));
    }
}

\end{lstlisting}

\subsubsection{Test funzione che controlla i campi di un ristorante.}

\begin{lstlisting}[language = Java , frame = trBL , firstnumber = last , escapeinside={(*@}{@*)}]
    public class getRestaurantFiedlsErrorsTest {

    public ArrayList<Integer> codici_errore = new ArrayList<Integer>();


    // L'ARRAYLIST DEVE ESSERE PULITO OGNI VOLTA CHE VIENE CONCLUSO UN CASO DI TEST
    @AfterEach
    public void clearArrayList(){
        codici_errore.clear();
    }

    @Test
    public void testgetRestaurantFieldsError(){
        ArrayList<Integer> actualErrors = getRestaurantFieldsErrors("Ristorante Test", "10", "Via Roma 1", "0123456789");
        assertEquals(codici_errore, actualErrors); //Funziona poiche non ci sono codici di errore: l'ArrayList risulta vuoto
    }

    @Test
    public void testNomeCampoTroppoCorto() {
        codici_errore.add(1);
        ArrayList<Integer> actualErrors = getRestaurantFieldsErrors("a", "10", "Via Roma 1", "0123456789");
        assertEquals(codici_errore, actualErrors);
    }

    @Test
    public void testCampoCopertiFuoriRange(){
        codici_errore.add(4);
        assertAll(
                () -> assertEquals(codici_errore, getRestaurantFieldsErrors("Ristorante Test", "1", "Via Roma 1", "0123456789")),
                () -> assertEquals(codici_errore, getRestaurantFieldsErrors("Ristorante Test", "100000", "Via Roma 1", "0123456789"))
        );
    }

    @Test
    public void testCampoCopertiNonValido(){
        codici_errore.add(9);
        assertAll(
                () -> assertEquals(codici_errore, getRestaurantFieldsErrors("Ristorante Test", "dieci", "Via Roma 1", "0123456789")),
                () -> assertEquals(codici_errore, getRestaurantFieldsErrors("Ristorante Test", "10a", "Via Roma 1", "0123456789"))
        );
    }


    @Test
    public void testLocazioneCampoTroppoCorto() {
        codici_errore.add(6);
        ArrayList<Integer> actualErrors = getRestaurantFieldsErrors("Ristorante Test", "10", "Via", "0123456789");
        assertEquals(codici_errore, actualErrors);
    }

    @Test
    public void testLocazioneNonValido() {
        codici_errore.add(10);
        ArrayList<Integer> actualErrors = getRestaurantFieldsErrors("Ristorante Test", "10", "Via_Napoli?", "0123456789");
        assertEquals(codici_errore, actualErrors);
    }


    @Test
    public void testNumeroTelefonoCampoTroppoCorto() {
        codici_errore.add(8);
        ArrayList<Integer> actualErrors = getRestaurantFieldsErrors("Ristorante Test", "10", "Via Roma 1", "12345678");
        assertEquals(codici_errore, actualErrors);
    }

    @Test
    public void testTuttiICampiVuoti() {
        codici_errore.add(2);
        codici_errore.add(3);
        codici_errore.add(5);
        codici_errore.add(7);
        ArrayList<Integer> actualErrors = getRestaurantFieldsErrors("", "", "", "");
        assertEquals(codici_errore, actualErrors);
    }
}
\end{lstlisting}

\subsubsection{Testing delle funzione media (in statistiche).}
\begin{lstlisting}[language = Java , frame = trBL , firstnumber = last , escapeinside={(*@}{@*)}]
    public class mediaTest {

    StatisticsActivityMock mock;

    @Before
    public void setUp(){
        mock = new StatisticsActivityMock();
    }

    //TESTING BLACK BOX

    @Test
    public void testMedia(){
        float result = mock.media(17, 50f);
        assertEquals(2.94f, result, 0.001f);
    }
    @Test
    public void testGiornoNegativoIncassoPositivo(){
        assertThrows(IllegalArgumentException.class,
                () -> mock.media(-3, 50.06f)
        );
    }
    @Test
    public void testGiornoPositivoIncassoNegativo(){
        assertThrows(IllegalArgumentException.class,
                () -> mock.media(3, -50.06f)
        );
    }
    @Test
    public void testGiornoEIncassoNegativi(){
        assertThrows(IllegalArgumentException.class,
                () -> mock.media(-3, -50.06f)
        );
    }

    @Test
    public void testGiornoZero(){
        assertAll(
                () ->  assertThrows(ArithmeticException.class,
                        () -> mock.media(0, 50.06f)
                ),
                () ->  assertThrows(ArithmeticException.class,
                        () -> mock.media(0, -50.06f)
                )
        );
    }

    //TESTING WHITE BOX

    @Test
    public void testGiornoNull(){
        Integer giorno = null;
        assertAll(
                () ->   assertThrows(NullPointerException.class,
                        () -> mock.media(giorno, 0.30f)
                ),
                () ->   assertThrows(NullPointerException.class,
                        () -> mock.media(giorno, -0.30f)
                )
        );
    }
    @Test
    public void testIncassoNull(){
        Float incasso = null;
        assertAll(
                () ->   assertThrows(NullPointerException.class,
                        () -> mock.media(0, incasso)
                ),
                () ->   assertThrows(NullPointerException.class,
                        () -> mock.media(3, incasso)
                ),
                () ->   assertThrows(NullPointerException.class,
                        () -> mock.media(-3, incasso)
                )
        );
    }

    @Test
    public void testCampiNull(){
        Integer giorno = null;
        Float incasso = null;
        assertThrows(NullPointerException.class,
                () -> mock.media(giorno, incasso)
        );
    }


}

\end{lstlisting}


