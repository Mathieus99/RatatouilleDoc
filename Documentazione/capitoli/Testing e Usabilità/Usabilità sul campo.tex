\subsection{Usabilità sul campo}
    \subsubsection{Premessa}
        \begin{flushleft}
        Qualunque sia la tecnica utilizzata, i test con gli utenti sono indispensabili. Infatti, le cause delle difficoltà incontrate
        dagli utenti possono essere moltissime. Analizzare un sistema “a tavolino”, come nelle valutazioni euristiche, anche se
        può permetterci d'individuare numerosi difetti, non è mai sufficiente. I problemi possono essere nascosti e verificarsi
        soltanto con certi utenti, in relazione alla loro esperienza o formazione. Cose ovvie per chi già conosce il sistema o
        sistemi analoghi possono rivelarsi difficoltà insormontabili per utenti meno esperti. Un test di usabilità ben condotto
        mette subito in evidenza queste difficoltà. La necessità del coinvolgimento degli utenti è affermata con chiarezza dalla
        stessa ISO 13407: "
            \emph{La valutazione condotta soltanto da esperti, senza il coinvolgimento degli utenti, può essere veloce ed
            economica, e permettere di identificare i problemi maggiori, ma non basta a garantire il successo di un
            sistema interattivo. La valutazione basata sul coinvolgimento degli utenti permette di ottenere utili indicazioni
            in ogni fase della progettazione. Nelle fasi iniziali, gli utenti possono essere coinvolti nella valutazione di
            scenari d’uso, semplici mock-up cartacei o prototipi parziali. Quando le soluzioni di progetto sono più
            sviluppate, le valutazioni che coinvolgono l’utente si basano su versioni del sistema progressivamente più
            complete e concrete. Può anche essere utile una valutazione cooperativa, in cui il valutatore discute con
            l’utente i problemi rilevati.}"
        \end{flushleft}
    
    \subsubsection{Presentazione degli utenti}
        \begin{flushleft}
            Per svolgere questo tipo di test importantissimi a prodotto finito abbiamo assunto che la nostra app fosse in versione "beta" e
            distribuito a una cerchia ristretta e selezionata di persone l'applicativo.
            Per avere un'idea dei test di valutazione dell'usabilià sul campo condotti riportiamo, tramite degli alias, quelle che sono state le
            "interviste" alle vere persone utilizzatrici.
            Gli intervistati sono \textbf{\emph{Luigi, Biagio, Matteo e Massimo:}}
            \\\emph{Luigi}, famoso imprenditore, voleva aprire un ristorante 3.0 in una zona turistica del proprio paese, e si è subito messo alla ricerca di un teami
            in grado di sostenerlo nella sua missione. 
            \\\emph{Matteo}, conosciuto per essere il "capo" che chiunque desidera. Sa gestire gruppi di persone, è bravo a comunicare e a segnalare le criticà in ambienti lavorativi.
            \\\emph{Biagio}, gran lavoratore, ama darsi da fare e sfruttare il massimo dalle tecnlogie a disposizione per offrire un servizio sempre di qualità.
            Anni e anni di esperienza come cameriere in hotel e ristoranti fanno di lui la persona perfetta per far parte del team di Luigi.
            \\\emph{Massimo}, appena diciottenne, si è affacciato al mondo del lavoro e cerca la sua prima esperienza come addetto alla cucina. Scopre che il risotrante di Luigi cerca giovani in gamba!
        
        \end{flushleft}

    \subsubsection{Il confronto con gli utenti}
        \begin{flushleft}
            Abbiamo dotato i dispositivi Android del personale del nuovissimo ristorante di Luigi della nostra applicazione Ratatuill23,
            e abbiamo monitorato per due giorni i risultati.

            \textbf{Domanda n.1:} Luigi, è stato difficile registrarti e registrare il tuo ristorante nell'app?\\
            \emph{Per nulla, anzi, la procedura di registrazione era guidata e un simpatico topolino mi ha guidato tra i vari controlli. Ho potuto abilitare l'opzione "zona turistica", ma a cosa serve?}\\
            \vspace{0.2cm}

            \textbf{Domanda n.2:} Biagio, com'è stato prendere il tuo primo ordine da cellulare?\\
            \emph{Molto seplice devo dire, non ho subito capito come rimuovere gli elementi dall'ordine, ma poi ho letto un'indicazione e ora so farlo! Inoltre improvvisamente il telefono ha suonato, era un avviso da Luigi!}\\
            \vspace{0.2cm}
            \newpage
            \textbf{Domanda n.3:} Massimo, come ti sei trovato con la nostra app?\\
            \emph{Effettivamente l'app è minimale e semplice da usare, in alcuni tratti forse un pochino troppo. In ogni caso, una notifica mi ha avvisato non appena è stato registrato un ordine.}
            
            \vspace{0.2cm}
            \textbf{Domanda n.4:} Matteo, come ti sei trovato con la nostra app?\\
            \emph{Come supervisore della mia sala, ad un certo punto ho avuto necessità di avvisare tutti i camerieri di un problema: grazie alla funzione avviso, ci sono riuscito facilmente.}
       
            \vspace{0.2cm}
            \textbf{Domanda n.5:} Cosa migliorereste nella nostra applicazione?\\
            \emph{Luigi: Avrei effettuato una scelta di colori diversa\\Biagio: Forse avrei reso più intuitivo cancellare i piatti da un ordine.\\Massimo:Non cambierei nulla, fa quello che deve fare, offrendo tutte le funzioni necessarie a una cucina.\\Matteo: Sarebbe da incrementare la velocità di risposta in alcuni casi}

            \vspace{0.2cm}
            Emerge da queste mini interviste un indice di gradimento dell'app di circa il 60/70\%, da confrontare e confermare con l'uso nel tempo dell'applicativo.



        \end{flushleft}

    \subsubsection{Valutazione finale}
        \begin{flushleft}
            Abbiamo notato che i feedback positivi sono stati maggiorni rispetto a quelli negativi e ciò rende il team entusiasta del proprio lavoro. Tra quelli negativi, notiamo che sono tutti risolvibili grazie alla scalabilità delle soluzioni tecnlogiche adottate e grazie alla possibilità di garantire aggiornamenti futuri in qualsiasi momento e su qualsiasi fronte del sistema.
        \end{flushleft}