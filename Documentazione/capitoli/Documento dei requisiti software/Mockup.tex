\subsection{Mock-up dell'applicazione}
    \begin{flushleft}
       
        L'interfaccia utente  ha il compito di “filtrare” la complessità, presentando all'utente un'immagine semplificata del prodotto, e congruente con i compiti che egli deve
        svolgere. Una buona interfaccia non solo nasconde la complessità interna del sistema, ma ne riduce la
        complessità funzionale, mettendo a disposizione dell'utente funzioni di più alto livello, in grado di effettuare compiti
        complessi con un grado di automatismo maggiore. (da "Facile da Usare")\\
        Qui vengono presentati i mockup relativi all'applicazione.

         \begin{figure}[H]
          \centering
          \includegraphics[scale=0.5]{assets/immagini varie/interfaccia utente.png}
          \caption{L'interfaccia Utente}\label{fig:interfaccia utente}
        \end{figure}

    \end{flushleft}

    \subsubsection{Homepage dell'applicazione}
        \begin{figure}[H]
          \centering
          \includegraphics[scale=0.5]{assets/diagrammi/Mockup/Mockup_Homepage.png}
          \caption{\textbf{M01}: Homepage dell'applicazione}
          \label{fig:Mockup_Homepage}
        \end{figure}

        \newpage

        \begin{flushleft}
            \textbf{ID} \ \Large{ \emph{\textbf{M01}}}\\
        \end{flushleft}

        \textbf{Componenti}:

        \begin{center}
          \begin{tblr}{hlines = {0.9pt}, vlines = {0.9pt}, row{1} = {marroneApp!60}, colspec = {X[c]X[c]X[c]}, width = \textwidth}
            \textbf{Tipo}  &   \textbf{Nome} & \textbf{Funzione} \\
            Bottone        &   ACCEDI        & Quando cliccato porta alla schermata  \emph{\textbf{M02}} \\
            Bottone        &   REGISTRATI   & Quando cliccato porta alla schermata  \emph{\textbf{M03}}  \\
          \end{tblr}
        \end{center}
        
        \newpage

        \subsubsection{Schermata di accesso nel sistema}
            \begin{figure}[H]
                \centering
                \includegraphics[scale=0.5]{assets/diagrammi/Mockup/Mockup_Accesso.png}
                \caption{\textbf{M02}: Schermata di accesso nel sistema}\label{fig:Mockup_Login}
            \end{figure}
            \begin{flushleft}
                \textbf{ID} \ \Large{ \emph{\textbf{M02}}}\\
            \end{flushleft}

            \textbf{Componenti}:

            \begin{center}
              \begin{tblr}{hlines = {0.9pt}, vlines = {0.9pt}, row{1} = {marroneApp!60}, colspec = {X[c]X[c]X[c]}, width = \textwidth}
                \textbf{Tipo}   &   \textbf{Nome}   &   \textbf{Funzione} \\
                Edit Text       &   EMAIL &   Permette l'inserimento dell'email dell'utente \\
                Edit Text       &   PASSWORD  &  Permette l'inserimento della password dell'utente  \\
                Bottone         &   ENTRA   & Quando cliccato porta alla schermata  \emph{\textbf{M04}} se admin,  \emph{\textbf{M09}} se supervisore,  \emph{\textbf{M11}} se cameriere,  \emph{\textbf{M12}} se account della cucina \\
              \end{tblr}
            \end{center}

        \newpage
        \subsubsection{Schermata di registrazione nel sistema}
        \begin{figure}[H]
          \centering
          \includegraphics[scale=0.25]{assets/diagrammi/Mockup/Mockup_Register.png}
          \caption{\textbf{M03}: Schermata di registrazione nel sistema}\label{fig:Mockup_Register}
        \end{figure}

        \begin{flushleft}
          \textbf{ID} \ \Large{ \emph{\textbf{M03}}}\\
          \large{ \emph{Nota}: la schermata di registrazione è valida solo per gli admin proprietari dei ristoranti, in quanto sono poi loro a registrare i dipendenti.}\\
        \end{flushleft}

        \textbf{Componenti}:

            \begin{center}
              \begin{longtblr}{hlines = {0.9pt}, vlines = {0.9pt}, row{1} = {marroneApp!60}, colspec = {X[c]X[c]X[c]}, width = \textwidth, rowhead=1}
                \textbf{Tipo}   &   \textbf{Nome}   &   \textbf{Funzione} \\
                Edit Text    &   NOME    &   Permette di inserire il nome dell'admin \\
                Edit Text & COGNOME   &  Permette di inserire il cognome dell'admin \\
                Edit Text    &   PASSWORD    &   Permette di inserire una password per l'admin \\
                Edit Text    &   EMAIL   &   Permette di inserire l'email dell'admin \\
                Edit Text    & CODICE FISCALE    & Permette di inserire il codice fiscale dell'admin \\
                Edit Text    &   P.IVA   & Permette di inserire la partita IVA dell'admin \\
                Bottone &   REGISTRATI  & Quando cliccato riporta alla schermata  \emph{\textbf{M01}} per permettere l'accesso \\
              \end{longtblr}
            \end{center}
        \newpage
        \subsubsection{Schermata home per gli admin}
        \begin{figure}[H]
            \centering
            \includegraphics[scale=0.35]{assets/diagrammi/Mockup/Mockup_AdminDash.png}
            \caption{\textbf{M04}: Schermata home per gli admin}\label{fig:Mockup_AdminDashboard}
        \end{figure}
        \begin{flushleft}
            \textbf{ID} \ \Large{ \emph{\textbf{M04}}} \\
        \end{flushleft}
        \textbf{Componenti}:

            \begin{center}
                \begin{longtblr}{hlines = {0.9pt}, vlines = {0.9pt}, row{1} = {marroneApp!60}, colspec = {X[c]X[c]X[c]}, width = \textwidth, rowhead=1}
                \textbf{Tipo}   &   \textbf{Nome}   &   \textbf{Funzione} \\
                ScrollView &   I TUOI RISTORANTI    &   Visualizza ed eventualmente permette la modifica dei ristoranti registrati\\
                Bottone    &   AGGIUNGI RISTORANTE  &   Quando cliccato porta alla schermata  \emph{\textbf{M05}} \\
                Bottone    &   MODIFICA   &   Quando cliccato porta alla schermata  \emph{\textbf{M06}} \\
                Bottone    &   PROFILO    &   Quando cliccato porta alla schermata  \emph{\textbf{M13}} \\
                Bottone    &   ESCI       &   Quando cliccato mostra la schermata  \emph{\textbf{M10}} \\
                Bottone    &   STATISTICHE &  Quando cliccato porta alla schermata  \emph{\textbf{M20}} \\
                \end{longtblr}
            \end{center}
        \newpage

        \subsubsection{Schermata di registrazione di un nuovo ristorante}
        \begin{figure}[H]
            \centering
            \includegraphics[scale=0.35]{assets/diagrammi/Mockup/Mockup_SaveResturant.png}
            \caption{\textbf{M05}: Schermata di registrazione di un nuovo ristorante}\label{fig:Mockup_AddResturant}
        \end{figure}
        \begin{flushleft}
            \textbf{ID} \ \Large{ \emph{\textbf{M05}}}
        \end{flushleft}
        \textbf{Componenti}:

        \begin{center}
          \begin{tblr}{hlines = {0.9pt}, vlines = {0.9pt}, row{1} = {marroneApp!60}, colspec = {X[c]X[c]X[c]}, width = \textwidth}
            \textbf{Tipo}  &   \textbf{Nome}  & \textbf{Funzione} \\
            Edit Text      &   NOME           & Permette di inserire il nome del nuovo ristorante\\
            Edit Text      &   LOCALITA'      & Permette di inserire la località del nuovo ristorante\\
            Edit Text      &   COPERTI        & Permette di inserire il n° dei coperti del nuovo ristorante\\
            Edit Text      &   NUMERO DI TELEFONO & Permette di inserire il contatto telefonico del ristorante \\
            Bottone        &   SALVA          & Quando cliccato salva il nuovo ristorante nel database e torna alla schermata  \emph{\textbf{M04}} \\
          \end{tblr}
        \end{center}
        \newpage
        \subsubsection{Schermata di modifica di un ristorante}
        \begin{figure}[H]
            \centering
            \includegraphics[scale=0.4]{assets/diagrammi/Mockup/Mockup_ResturantDash.png}
            \caption{\textbf{M06}: Schermata di modifica di un ristorante}\label{fig:Mockup_ResturantManager}
        \end{figure}
        \begin{flushleft}
            \textbf{ID} \ \Large{ \emph{\textbf{M06}}}
        \end{flushleft}

        \textbf{Componenti}:

          \begin{center}
            \begin{tblr}{hlines = {0.9pt}, vlines = {0.9pt}, row{1} = {marroneApp!60}, colspec = {X[c]X[c]X[c]}, width = \textwidth}
              \textbf{Tipo}  &   \textbf{Nome}  & \textbf{Funzione} \\
              Bottone   &   AGGIUNGI DIPENDENTE &   Quando cliccato porta alla schermata  \emph{\textbf{M07}}\\
              Bottone   &   GESTISCI MENU &   Quando cliccato porta alla schermata  \emph{\textbf{M08}}\\
              ScrollView  & IL TUO PERSONALE  & Mostra il personale che lavora nel ristorante, dove se si tiene premuto il bottone  \emph{Elimina} viene eliminato l'account del dipendente scelto dal ristorante \\
              Bottone   &   CREA AVVISO   &   Quando cliccato mostra la schermata  \emph{\textbf{M17}} \\  
            \end{tblr}
          \end{center}

        \newpage

        \subsubsection{Schermata di registrazione di un nuovo dipendente}
        \begin{figure}[H]
            \centering
            \includegraphics[scale=0.35]{assets/diagrammi/Mockup/Mockup_SaveWorker.png}
            \caption{\textbf{M07}: Schermata di registrazione di un nuovo dipendente}\label{fig:Mockup_SaveWaiter}
        \end{figure}
        \begin{flushleft}
            \textbf{ID} \ \Large{ \emph{\textbf{M07}}}
        \end{flushleft}
        \textbf{Componenti}:

        \begin{center}
          \begin{tblr}{hlines = {0.9pt}, vlines = {0.9pt}, row{1} = {marroneApp!60}, colspec = {X[c]X[c]X[c]}, width = \textwidth}
            \textbf{Tipo}   &   \textbf{Nome}   &   \textbf{Funzione} \\
            Edit Text   &   NOME    &   Permette di inserire il nome del nuovo dipendente\\
            Edit Text   &   COGNOME   &   Permette di inserire il cognome del nuovo dipendente\\
            Edit Text   &   EMAIL   & Permette di inserire l'email del nuovo dipendente\\
            Edit Text   &   CODICE FISCALE    &   Permette di inserire il codice fiscale del nuovo dipendente \\
            Spinner &   {cameriere\\ (default)}    &   Permette di inserire il ruolo del nuovo dipendente \\
            Bottone &   REGISTRA    &   Quando cliccato, se tutti i dati sono corretti, riporta alla schermata  \emph{\textbf{M04}} registrando il nuovo dipendente \\
          \end{tblr}
        \end{center}

        \newpage

        \subsubsection{Schermata di gestione del menù}
        \begin{figure}[H]
            \centering
            \includegraphics[scale=0.35]{assets/diagrammi/Mockup/Mockup_MenuManager.png}
            \caption{\textbf{M08}: Schermata di gestione del menù}\label{fig:Mockup_MenuManager}
        \end{figure}
        \begin{flushleft}
            \textbf{ID} \ \Large{ \emph{\textbf{M08}}}
        \end{flushleft}
        \textbf{Componenti}:

        \begin{center}
          \begin{tblr}{hlines = {0.9pt}, vlines = {0.9pt}, row{1} = {marroneApp!60}, colspec = {X[c]X[c]X[c]}, width = \textwidth}
            \textbf{Tipo}   &   \textbf{Nome}   &   \textbf{Funzione} \\
            Bottone   &   AGGIUNGI PRODOTTO    &   Quando cliccato porta alla schermata  \emph{\textbf{M18}}\\
            Edit Text   &   SCARICA MENU   &   Permette di generare il menù e salvarlo sul dispositivo in formato PDF\\
            Bottone   &   CREA MENU       &   Quando cliccato permette di creare un menu (se non è gia stato creato) \\
            Bottone   &   CANCELLA MENU   &   Permette di cancellare il menu del ristorante (se esistente) \\
          \end{tblr}
        \end{center}

        \newpage

        \subsubsection{Schermata home per i supervisori}
          \begin{figure}[H]
              \centering
              \includegraphics[scale=0.35]{assets/diagrammi/Mockup/Mockup_HypervisorDash.png}
              \caption*{\textbf{M09}: Schermata home dei supervisori}\label{fig:Mockup_HypervisorDash}
          \end{figure}

          \begin{flushleft}
              \textbf{ID}   \ \Large{ \emph{\textbf{M09}}}
          \end{flushleft}

          \textbf{Componenti}:

          \begin{center}
              \begin{tblr}{hlines = {0.9pt}, vlines = {0.9pt}, row{1} = {marroneApp!60}, colspec = {X[c]X[c]X[c]}, width = \textwidth}
                \textbf{Tipo}   &   \textbf{Nome}   &   \textbf{Funzione} \\
                Bottone   &   PROFILO               &   Quando cliccato porta alla schermata  \emph{\textbf{M13}}  \\
                Bottone   &   NUOVO AVVISO          &   Quando cliccato porta alla schermata  \emph{\textbf{M21}}  \\ 
                Bottone   &   PENDING ORDERS        &   Quando cliccato porta alla schermata  \emph{\textbf{M12}}  \\ 
                Bottone   &   ESCI                  &   Quando cliccato porta alla schermata  \emph{\textbf{M10}}  \\
                Bottom Navigation Bar & BARRA NAVIGAZIONE   &   Permette di visualizzare le notifiche cliccando su  \emph{Notifiche} che porterà alla schermata  \emph{\textbf{M21}} e di tornare indietro cliccando su  \emph{Indietro} \\
              \end{tblr}
          \end{center}

        \newpage

        \subsubsection{Schermata di conferma di uscita}
          \begin{figure}[H]
            \centering
            \includegraphics[scale=0.5]{assets/diagrammi/Mockup/Mockup_ExitDialog.png}
            \caption*{\textbf{M10}: Schermata di conferma logout}\label{fig:Mockup_ExitDialog}
          \end{figure}

          \begin{flushleft}
            \textbf{ID}   \ \Large{ \emph{\textbf{M10}}}
          \end{flushleft}

          \textbf{Componenti}:

          \begin{center}
            \begin{tblr}{hlines = {0.9pt}, vlines = {0.9pt}, row{1} = {marroneApp!60}, colspec = {X[c]X[c]X[c]}, width = \textwidth}
              \textbf{Tipo}   &   \textbf{Nome}   &   \textbf{Funzione} \\
              Bottone         &   SI      &   Quando cliccato porta alla schermata  \emph{\textbf{M02}} \\
              Bottone         &   NO      &   Quando cliccato resta nella schermata corrente \\
            \end{tblr}
          \end{center}
        
        \newpage

        \subsubsection{Schermata home per i camerieri}
          \begin{figure}[H]
            \centering
            \includegraphics[scale=0.5]{assets/diagrammi/Mockup/Mockup_WaiterDash.png}
            \caption*{\textbf{M11}: Schermata home per i camerieri}
            \label{fig:Mockup_WaiterDash}
          \end{figure}

          \begin{flushleft}
            \textbf{ID}   \ \Large{ \emph{\textbf{M11}}}
          \end{flushleft}

          \textbf{Componenti}:
          
          \begin{center}
            \begin{tblr}{hlines = {0.9pt}, vlines = {0.9pt}, row{1} = {marroneApp!60}, colspec = {X[c]X[c]X[c]}, width = \textwidth}
              \textbf{Tipo}   &   \textbf{Nome}   &   \textbf{Funzione} \\
              Bottone     &   NUOVO ORDINE    &   Quando cliccato porta alla schermata  \emph{\textbf{M19}} \\
              Bottone     &   STATO ORDINE    &   Quando cliccato porta alla schermata  \emph{\textbf{M12}} \\
              Bottom Navigation Bar & BARRA NAVIGAZIONE   &   Permette di visualizzare le notifiche cliccando su  \emph{Notifiche} che porterà alla schermata  \emph{\textbf{M21}} e di tornare indietro cliccando su  \emph{Indietro} \\
            \end{tblr}
          \end{center}

        \newpage

        \subsubsection{Schermata di visualizzazione dello stato degli ordini}

          \begin{figure}[H]
            \centering
            \includegraphics[scale=0.5]{assets/diagrammi/Mockup/Mockup_OrderStatus.png}
            \caption*{\textbf{M12}: Schermata home per i camerieri}
            \label{fig:Mockup_WaiterDash}
          \end{figure}

            \begin{flushleft}
              \textbf{ID}   \ \Large{ \emph{\textbf{M12}}}
            \end{flushleft}
  
            \textbf{Componenti}:
            
            \begin{center}
              \begin{tblr}{hlines = {0.9pt}, vlines = {0.9pt}, row{1} = {marroneApp!60}, colspec = {X[c]X[c]X[c]}, width = \textwidth}
                \textbf{Tipo}   &   \textbf{Nome}   &   \textbf{Funzione} \\
                RecyclerView     &  ORDINI    &   Permette di visualizzare la lista degli ordini, dove per i camerieri vi è la possibilità di sollecitare la cucina su un ordine e per la cucina di evaderlo eliminandolo  \\
                Bottom Navigation Bar & BARRA NAVIGAZIONE   &   Permette di visualizzare le notifiche cliccando su  \emph{Notifiche} che porterà alla schermata  \emph{\textbf{M21}} e di tornare indietro cliccando su  \emph{Indietro} \\ 
              \end{tblr}
            \end{center}

            \newpage

            \subsubsection{Schermata di visualizzazione del profilo}
              \begin{figure}[H]
                \centering
                \includegraphics[scale=0.5]{assets/diagrammi/Mockup/Mockup_Profile.png}
                \caption*{\textbf{M13}: Schermata visualizzazione profilo}\label{fig:Mockup_Profile}
              \end{figure}
    
              \begin{flushleft}
                \textbf{ID}   \ \Large{ \emph{\textbf{M13}}}
              \end{flushleft}
    
              \textbf{Componenti}:
              
              \begin{center}
                \begin{tblr}{hlines = {0.9pt}, vlines = {0.9pt}, row{1} = {marroneApp!60}, colspec = {X[c]X[c]X[c]}, width = \textwidth}
                  \textbf{Tipo}   &   \textbf{Nome}   &   \textbf{Funzione} \\
                  Bottone     &   CAMBIA PASSWORD   &   Quando cliccato mostra la schermata  \emph{\textbf{M14}}  \\
                  Bottone     &   CAMBIA EMAIL   &   Quando cliccato mostra la schermata  \emph{\textbf{M15}}  \\    
                \end{tblr}
              \end{center}

              \newpage

              \subsubsection{Schermata di cambio password per gli admin}
                \begin{figure}[H]
                  \centering
                  \includegraphics[scale=0.5]{assets/diagrammi/Mockup/Mockup_AdminChangePass.png}
                  \caption*{\textbf{M14}: Schermata cambio password admin}\label{fig:Mockup_AdminChangePass}
                \end{figure}
      
                \begin{flushleft}
                  \textbf{ID}   \ \Large{ \emph{\textbf{M14}}}
                \end{flushleft}
      
                \textbf{Componenti}:
                
                \begin{center}
                  \begin{tblr}{hlines = {0.9pt}, vlines = {0.9pt}, row{1} = {marroneApp!60}, colspec = {X[c]X[c]X[c]}, width = \textwidth}
                    \textbf{Tipo}   &   \textbf{Nome}   &   \textbf{Funzione} \\
                    Edit Text     &   NUOVA PASSWORD    &   Permette di inserire la nuova password per l'account dell'admin   \\
                    Bottone     &   ANNULLA   &   Quando cliccato torna alla schermata  \emph{\textbf{M13}} senza effettuare modifiche  \\
                    Bottone     &   SALVA   &   Quando cliccato torna alla schermata  \emph{\textbf{M13}} modificando la password dell'account  \\
                  \end{tblr}
                \end{center}

                \newpage

                \subsubsection{Schermata di cambio email per gli admin}
                  \begin{figure}[H]
                    \centering
                    \includegraphics[scale=0.5]{assets/diagrammi/Mockup/Mockup_AdminChangeMail.png}
                    \caption*{\textbf{M15}: Schermata cambio email admin}\label{fig:Mockup_AdminChangeMail}
                  \end{figure}
        
                  \begin{flushleft}
                    \textbf{ID}   \ \Large{ \emph{\textbf{M15}}}
                  \end{flushleft}
        
                  \textbf{Componenti}:
                  
                  \begin{center}
                    \begin{tblr}{hlines = {0.9pt}, vlines = {0.9pt}, row{1} = {marroneApp!60}, colspec = {X[c]X[c]X[c]}, width = \textwidth}
                      \textbf{Tipo}   &   \textbf{Nome}   &   \textbf{Funzione} \\
                      Edit Text   &   NUOVA EMAIL   &   Permette di inserire la nuova email per l'account dell'admin  \\
                      Bottone     &   ANNULLA   &   Quando cliccato torna alla schermata  \emph{\textbf{M13}} senza effettuare modifiche  \\
                      Bottone     &   SALVA   &   Quando cliccato torna alla schermata  \emph{\textbf{M13}} modificando l'email dell'account  \\
                    \end{tblr}
                  \end{center}

                \newpage

                \subsubsection{Schermata di cambio password per i dipendenti al primo accesso}
                    \begin{figure}[H]
                      \centering
                      \includegraphics[scale=0.35]{assets/diagrammi/Mockup/Mockup_WorkerChangePass.png}
                      \caption*{\textbf{M16}: Schermata cambio password dipendenti}\label{fig:Mockup_WorkerChangePass}
                    \end{figure}
          
                    \begin{flushleft}
                      \textbf{ID}   \ \Large{ \emph{\textbf{M16}}}
                    \end{flushleft}
          
                    \textbf{Componenti}:
                    
                    \begin{center}
                      \begin{tblr}{hlines = {0.9pt}, vlines = {0.9pt}, row{1} = {marroneApp!60}, colspec = {X[c]X[c]X[c]}, width = \textwidth}
                        \textbf{Tipo}   &   \textbf{Nome}   &   \textbf{Funzione} \\
                        Edit Text   &   NUOVA PASSWORD   &   Permette di inserire la nuova password per l'account del dipendente  \\
                        Bottone     &   ANNULLA   &   Quando cliccato torna alla schermata  \emph{\textbf{M02}} senza effettuare modifiche  \\
                        Bottone     &   SALVA   &   Quando cliccato porta alla schermata  \emph{\textbf{M09}} se è supervisore,  \emph{\textbf{M11}} se è cameriere,  \emph{\textbf{M12}} se è un account di cucina, modificando l'email dell'account  \\
                      \end{tblr}
                    \end{center}

                  \newpage

                  \subsubsection{Schermata di creazione degli avvisi}
                      \begin{figure}[H]
                        \centering
                        \includegraphics[scale=0.5]{assets/diagrammi/Mockup/Mockup_SaveAdv.png}
                        \caption*{\textbf{M17}: Schermata creazione avvisi}\label{fig:Mockup_SaveAdv}
                      \end{figure}
            
                      \begin{flushleft}
                        \textbf{ID}   \ \Large{ \emph{\textbf{M17}}}
                      \end{flushleft}
            
                      \textbf{Componenti}:

                      \begin{center}
                        \begin{tblr}{hlines = {0.9pt}, vlines = {0.9pt}, row{1} = {marroneApp!60}, colspec = {X[c]X[c]X[c]}, width = \textwidth}
                          \textbf{Tipo}   &   \textbf{Nome}   &   \textbf{Funzione} \\
                          Edit Text     &   AVVISO    &   Permette di inserire il testo dell'avviso \\
                          Bottone       &   INVIA      &   Quando cliccato torna alla schermata  \emph{\textbf{M06}} inviando l'avviso \\
                        \end{tblr}
                      \end{center}

                    \newpage

                    \subsubsection{Schermata di aggiunta piatti al menù}
                        \begin{figure}[H]
                          \centering
                          \includegraphics[scale=0.5]{assets/diagrammi/Mockup/Mockup_AddPlate.png}
                          \caption*{\textbf{M18}: Schermata aggiunta piatti al menù}\label{fig:Mockup_AddPlate}
                        \end{figure}
              
                        \begin{flushleft}
                          \textbf{ID}   \ \Large{ \emph{\textbf{M18}}}
                        \end{flushleft}
              
                        \textbf{Componenti}:
                        
                        \begin{center}
                          \begin{longtblr}{hlines = {0.9pt}, vlines = {0.9pt}, row{1} = {marroneApp!60}, colspec = {X[c]X[c]X[c]}, width = \textwidth, rowhead=1}
                            \textbf{Tipo}   &   \textbf{Nome}   &   \textbf{Funzione} \\
                            EditText        &   NOME PRODOTTO   &   Permette di inserire il titolo del prodotto da inserire \\
                            EditText        &   DESCRIZIONE     &   Permette di inserire la descrizione del prodotto da Inserire  \\
                            EditText        &   ALLERGENI       &   Permette di inserire gli allergeni contenuti nel prodotto \\
                            EditText        &   PREZZO          &   Permette di inserire il prezzo del prodotto \\
                            Spinner         &   TIPO            &   Permette di inserire il tipo di portata (antipasto, primo, ...) \\
                            Spinner         &   TIPO DI ALIMENTO &  Permette di inserire il tipo di alimento (carne, pesce, ...)  \\
                            Bottone         &   CERCA           &   Quando cliccato, ricerca il prodotto scritto nel database di OpenFoodFacts  \\
                            Bottone         &   OK              &   Quando cliccato, ritorna alla schermata  \emph{\textbf{M08}} salvando nel menu il prodotto \\
                            Bottone         &   ANNULLA         &   Quando cliccate, ritorna alla schermata  \emph{\textbf{M08}} senza salvare il prodotto \\  
                            Selettore       &   PRECONFEZIONATO &   Se selezionato, indica che l'oggetto è di tipo preconfezionato(bibita, ...) \\
                          \end{longtblr}
                        \end{center}

                      \newpage

                    \subsubsection{Schermata di aggiunta ordini}
                          \begin{figure}[H]
                            \centering
                            \includegraphics[scale=0.4]{assets/diagrammi/Mockup/Mockup_AddOrder.png}
                            \caption*{\textbf{M19}: Schermata aggiunta ordini}\label{fig:Mockup_WaiterDash}
                          \end{figure}
                
                          \begin{flushleft}
                            \textbf{ID}   \ \Large{ \emph{\textbf{M19}}}
                          \end{flushleft}
                
                          \textbf{Componenti}:
                          
                          \begin{center}
                            \begin{tblr}{hlines = {0.9pt}, vlines = {0.9pt}, row{1} = {marroneApp!60}, colspec = {X[c]X[c]X[c]}, width = \textwidth}
                              \textbf{Tipo}   &   \textbf{Nome}   &   \textbf{Funzione} \\
                              ScrollView      &   PIATTI    &   Visualizza la lista dei piatti del menù del ristorante, dove ogni piatto si può cliccare per aggiungerlo all'ordine \\
                              Bottone         &   SALVA     &   Quando cliccato porta alla schermata  \emph{\textbf{M11}} inviando l'ordine \\
                              Bottone         &   INDIETRO  &   Quando cliccato porta alla schermata  \emph{\textbf{M11}} senza effettuare nessuna azione \\
                            \end{tblr}
                          \end{center}
                    
                    \newpage

                    \subsubsection{Schermata di visualizzazione delle statistiche}
                      \begin{figure}[H]
                        \centering
                        \includegraphics[scale=0.5]{assets/diagrammi/Mockup/Mockup_Statistics.png}
                        \caption*{\textbf{M20}: Schermata visualizzazione statistiche}\label{fig:Mockup_Statistics}
                      \end{figure}

                      \begin{flushleft}
                        \textbf{ID}   \ \Large{ \emph{\textbf{M20}}}
                      \end{flushleft}

                      \textbf{Componenti}:

                      \begin{center}
                        \begin{longtblr}{hlines = {0.9pt}, vlines = {0.9pt}, row{1} = {marroneApp!60}, colspec = {X[c]X[c]X[c]}, width = \textwidth, rowhead=1}
                          \textbf{Tipo}   &   \textbf{Nome}   &   \textbf{Funzione} \\
                          Spinner         &   RISTORANTE             &   Permette di selezionare il ristorante di cui si vogliono visualizzazione le statistiche \\
                          ScrollView      &   GRAFICI STATISTICHE    &   Permette di visualizzare i grafici relativi alle statistiche di  \emph{Ristorante} \\
                        \end{longtblr}
                      \end{center}
                    
                    \newpage

                    \subsubsection{Schermata di visualizzazione delle notifiche}
                      \begin{figure}[H]
                        \centering
                        \includegraphics[scale=0.4]{assets/diagrammi/Mockup/Mockup_Notifications.png}
                        \caption*{\textbf{M21}: Schermata visualizzazione notifiche}\label{fig:Mockup_Notifications}
                      \end{figure}

                      \begin{flushleft}
                        \textbf{ID}   \ \Large{ \emph{\textbf{M21}}}
                      \end{flushleft}

                      \textbf{Componenti}:

                      \begin{center}
                        \begin{longtblr}{hlines = {0.9pt}, vlines = {0.9pt}, row{1} = {marroneApp!60}, colspec = {X[c]X[c]X[c]}, width = \textwidth, rowhead=1}
                          \textbf{Tipo}   &   \textbf{Nome}   &   \textbf{Funzione} \\
                          ScrollView      &   NOTIFICHE    &   Permette di visualizzare le varie notifiche ricevute dal proprio ristorante, inviate dall'admin o dal supervisore, contenenti: 
                                                                \emph{Nome}, 
                                                                \emph{Data e ora},
                                                                \emph{Messaggio}, 
                                                               che tramite swipe verso sinistra vengono segnati come letti ed eliminati dalla schermata \\
                        \end{longtblr}
                      \end{center}