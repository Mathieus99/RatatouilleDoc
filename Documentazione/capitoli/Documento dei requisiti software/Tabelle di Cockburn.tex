\subsubsection{Tabelle di Cockburn} %sezione 2.7
    \begin{flushleft}
        Vengono qui riportate le tabelle di cockburn dei casi d'uso  \emph{\textbf{Aggiungi ristorante}},  \emph{\textbf{Aggiungi piatto}},  \emph{\textbf{Prendi ordinazione}},  \emph{\textbf{Visualizza avvisi}}.
    \end{flushleft}

    \begin{center}
      \begin{longtblr}{hlines = {0.9pt}, vlines = {0.9pt}, row{1}={marroneApp!60},colspec = {X[c]X[c]X[c]X[c]}, width = \textwidth,  rowhead=1}
        \textbf{Use Case \#1} & \SetCell[c=3]{c} \textbf{Aggiunta ristorante} \\
        \textbf{Goal in Context} & \SetCell[c=3]{l}{Un admin (proprietario di uno o più ristoranti)\\ vuole aggiungere uno di questi nell'app Ratatuille.}\\

        \textbf{Precodition} & \SetCell[c=3]{l}{Il proprietario deve essere registrato e loggato nell'app come amministratore}\\

        \textbf{Success End Condition} & \SetCell[c=3]{l}{Il proprietario aggiunge correttamente il ristorante}\\

        \textbf{Failed End Condition}  & \SetCell[c=3]{l}{Il proprietario non riesce ad aggiungere il ristorante}\\

        \textbf{Primary Actor}  & \SetCell[c=3]{c}{Amministratore}\\
        \textbf{Trigger}  & \SetCell[c=3]{c}{Preme il pulsante  "Aggiungi "}\\

        \SetCell[r=5]{c}\textbf{Description}  & Step & UserAction & System\\
                                              & 1    & {L'amministratore preme sul tasto  \emph{ "AGGIUNGI RISTORANTE"}\\ sulla schermata \textbf{ \emph{M04}}} & \\
                                              & 2    &       & {Mostra la schermata \textbf{ \emph{M06}}}\\
                                              & 3    &  {Compila i campi necessari per la registrazione del proprio ristorante e preme sul tasto  "SALVA " per aggiungere il Ristorante}     & \\
                                              & 4    &       & {Ricarica la schermata \textbf{ \emph{M04}} aggiungendo nella lista dei ristoranti l'ultimo appena inserito} \\
        %% Extensions
        \SetCell[r=2]{c}{\textbf{Extension \#\ 1}\\ L'amministratore non fa nulla e torna indietro} & Step & UserAction & System\\
                                                                                                    & 4a   &  & Torna alla schermata \textbf{ \emph{M04}}\\

        \SetCell[r=2]{c}{\textbf{Extension \#2}\\ L'amministratore ha aggiunto un ristorante già presenta nella propria lista di ristoranti} & Step & UserAction & System\\*
                                                  & 4b   &  & {Mostra uno dei messaggi di errore della schermata \textbf{ \emph{{M0boh:}}} restando allo step \textbf{2} dello scenario principale}\\

        \SetCell[r=2]{c}{\textbf{Extension \#3}\\ L'amministratore ha lasciato uno o più campi vuoti}
                                                  & Step & UserAction & System\\
                                                  & 4c   &  & {Mostra uno o più dei messaggi di errore della schermata \textbf{ \emph{{M0boh}}} restando allo step \textbf{2} dello scenario principale}\\

        \SetCell[r=2]{c}{\textbf{Extension \#4}\\ L'amministratore ha aggiunto un nome troppo corto}
                                                  & Step & UserAction & System\\
                                                  & 4d   &  & {Mostra il messaggio di errore di fianco al campo  "Nome " della schermata \textbf{ \emph{{M0boh}}} restando allo step \textbf{2} dello scenario principale} \\

        \SetCell[r=2]{c}{\textbf{Extension \#5}\\ L'amministratore ha aggiunto un numero di coperti non valido}
                                                  & Step & UserAction & System\\
                                                  & 4e   &  & {Mostra il messaggio di errore di fianco al campo  "Numero di coperti'' della schermata \textbf{ \emph{{M0boh}}} restando allo step \textbf{2} dello scenario principale} \\

        \SetCell[r=2]{c}{\textbf{Extension \#6}\\ L'amministratore ha aggiunto un indirizzo troppo corto}
                                                  & Step & UserAction & System\\
                                                  & 4f   &  & {Mostra il messaggio di errore di fianco al campo  "Indirizzo " della schermata \textbf{ \emph{{M0boh}}} restando allo step \textbf{2} dello scenario principale} \\

        \SetCell[r=2]{c}{\textbf{Extension \#7}\\ L'amministratore ha aggiunto un numero di telefono non valido} 
                                                  & Step & UserAction & System\\*
                                                  & 4g   &  & {Mostra il messaggio di errore di fianco al campo  "Numero di telefono " della schermata \textbf{ \emph{{M0boh}}} restando allo step \textbf{2} dello scenario principale} \\

        %%% Subvariations
        \SetCell[r=4]{c}{\textbf{Subvariation \#1} \\ L'amministratore torna indietro completando solo parzialmente i campi}  & Step & UserAction & System\\
                                                                                                                             & 1 & & {Mostra la schermata \textbf{ \emph{M0boh}}}\\
                                                                                                                             & 2 & {Preme su  "SI "} & \\
                                                                                                                             & 3 & & {Torna alla schermata \textbf{ \emph{M04}}}\\

        \SetCell[r=4]{c}{\textbf{Subvariation \#2} \\ L'amministratore inizialmente vuole tornare indietro completando solo parzialmente i campi ma cambia idea}
                                                        & Step & UserAction & System\\
                                                        & 1 & & {Mostra la schermata \textbf{ \emph{M0boh}}}\\
                                                        & 2 & {Preme su  "NO "} & \\
                                                        & 3 & & {Torna alla schermata \textbf{ \emph{M06}} alo step \textbf{2} dello scenario principale}\\

      \end{longtblr}
    \end{center}

    
    
    \newpage
      
        \begin{center}
        \begin{longtblr}{hlines = {0.9pt}, vlines = {0.9pt}, row{1}={marroneApp!60}, colspec = {X[c]X[c]X[c]X[c]}, width = \textwidth}
          \textbf{Use Case \#2} & \SetCell[c=3]{c} \textbf{Prendere un ordine} \\
          \textbf{Goal in Context} & \SetCell[c=3]{l}{Un cameriere vuole prendere l'ordinazione di un tavolo}\\
        
          \textbf{Precodition} & \SetCell[c=3]{l}{Il cameriere deve essere registrato e loggato nell'app \\ e il ristorante deve avere un menu con dei piatti al suo interno}\\
        
          \textbf{Success End Condition} & \SetCell[c=3]{l}{Il cameriere prende correttamente un'ordinazione}\\
        
          \textbf{Failed End Condition}  & \SetCell[c=3]{l}{Il cameriere non prende l'ordinazione}\\
        
          \textbf{Primary Actor}  & \SetCell[c=3]{c}{Cameriere}\\
          \textbf{Trigger}  & \SetCell[c=3]{c}{Preme il pulsante  "NUOVO ORDINE "}\\
          
          \SetCell[r=5]{c}\textbf{Description}  & Step & UserAction & System\\
                                        & 1 & {Il cameriere preme il pulsante  \emph{ "NUOVO ORDINE "}\\ sulla schermata \textbf{ \emph{M0boh_1}}} & \\
                                        & 2 &  & {Mostra la schermata \textbf{ \emph{M0boh_2}} con i vari piatti del menu} \\
                                        & 3 & {Il cameriere seleziona il numero del tavolo, inserisce le portate all'interno dell'ordine e preme il pulsante  \emph{ "SALVA "}} & \\
                                        & 4 &  & {Torna alla schermata \textbf{ \emph{M0boh_2}}} \\
	   %% Extensions
	  \SetCell[r=2]{c}{\textbf{Extension \#\ 1}\\ Il cameriere non fa nulla e torna indietro} & Step & UserAction & System\\
                                                                                                    & 4a   &  & {Torna alla schermata \textbf{ \emph{M04}}}\\
        
        %%% Subvariations
        \SetCell[r=4]{c}{\textbf{Subvariation \#1} \\ Il cameriere torna indietro dopo aver aggiunto \\ dei piatti all'ordine senza salvare}  & Step & UserAction & System\\
                                                                                                                             & 1 & & {Mostra la schermata \textbf{ \emph{M0boh_3}}}\\
                                                                                                                             & 2 & {Preme su  "SI "} & \\
                                                                                                                             & 3 & & {Torna alla schermata \textbf{ \emph{M0boh_1}}}\\
        %%% Subvariations
        \SetCell[r=4]{c}{\textbf{Subvariation \#2} \\ Il cameriere inizialmente vuole tornare \\ indietro dopo aver aggiunto dei piatti all'ordine \\ senza salvare, ma cambia idea}  & Step & UserAction & System\\
                                                                                                                             & 1 & & {Mostra la schermata \textbf{\emph{M0boh_3}}}\\
                                                                                                                             & 2 & {Preme su  "NO "} & \\
                                                                                                                             & 3 & & {Torna alla schermata \textbf{ \emph{M0boh_2}} al punto 2 \\ dello scenario principale}\\
        \end{longtblr}
      \end{center}
        
    



        \newpage
        
        \begin{center}
        \begin{longtblr}{hlines = {0.9pt}, vlines = {0.9pt}, row{1}={marroneApp!60}, colspec = {X[c]X[c]X[c]X[c]}, width = \textwidth}
          \textbf{Use Case \#3} & \SetCell[c=3]{c} \textbf{Aggiungi piatto al menu} \\
          \textbf{Goal in Context} & \SetCell[c=3]{l}{Un amministratore vuole aggiungere un piatto al proprio menu}\\
            \textbf{Precodition} & \SetCell[c=3]{l}{Il proprietario deve essere registrato e loggato nell'app come amministratore, \\deve aver aggiunto almeno un ristorane e deve averne creato un menu}\\
          
            \textbf{Success End Condition} & \SetCell[c=3]{l}{L'amministratore aggiunge correttamente un piatto al suo menu}\\
          
            \textbf{Failed End Condition}  & \SetCell[c=3]{l}{L'amministratore non riesce ad aggiungere un piatto al suo ristorante}\\
          
            \textbf{Primary Actor}  & \SetCell[c=3]{c}{Amministratore}\\
            \textbf{Trigger}  & \SetCell[c=3]{c}{Preme il pulsante  \emph{ "Aggiungi prodotto "}}\\
            
            \SetCell[r=5]{c}\textbf{Description}  & Step & UserAction & System\\
                                          & 1 & {L'amministratore preme sul pulsante  \emph{ "Aggiungi prodotto "}\\ sulla schermata \textbf{ \emph{M0boh_1}}} & \\
                                          & 2 &  & {Mostra la schermata \textbf{ \emph{M0boh_2}}}\\
                                          & 3 & {L'amministratore compila il campi del piatto e \\ preme sul pulsante pulsante  \emph{ "Ok "}}      & \\
                                          & 4 &  & {Torna alla schermata \textbf{ \emph{M0boh_1}}}   \\
          
            \SetCell[r=2]{c}{\textbf{Extension \#\ 1}\\ L'amministratore non fa nulla e torna indietro}  & Step & UserAction & System\\
                                                         & 4a  &  & Torna alla schermata \textbf{ \emph{M0boh_1}}\\

            \SetCell[r=2]{c}{\textbf{Extension \#\ 2}\\ L'amministratore compila solo parzialmente\\ i campi e preme sul pulsante  \emph{ "Ok "}}  & Step & UserAction & System\\
                                                         & 4a  &  & Mostra l'errore nella schermata \textbf{ \emph{M0boh_2err}}\\
          
            \SetCell[r=6]{c}{\textbf{Subvariation \#\ 1}\\ L'amministratore vuole aggiungere un prodotto\\ preconfezionato al proprio menu}  & Step & UserAction & System\\
                                                  & 1 &  & {Mostra la schermata \textbf{ \emph{M0boh_2pre}}} \\
                                                  & 2 & {L'amministratore scrive il nome (completo o parziale)\\ di un prodotto e preme il pulsante  \emph{ "Cerca "} } &  \\
                                                  & 3 &  & {Mostra i prodotti corrispondenti alla ricerca} \\
                                                  & 4 & {L'amministratore seleziona il prodotto desiderato\\ e preme il pulsante  \emph{ "Ok "}} &  \\
                                                  & 5 &  & {Torna alla schermata \textbf{ \emph{M0boh_1}}}   \\
            %\textbf{Notes}  & \SetCell[c=3]{c}{\textcolor{red}{TODO}}\\
          \end{longtblr}
      \end{center}



      \newpage

        
        \begin{center}
        \begin{longtblr}{hlines = {0.9pt}, vlines = {0.9pt}, row{1}={marroneApp!60}, colspec = {X[c]X[c]X[c]X[c]}, width = \textwidth}
          \textbf{Use Case \#4} & \SetCell[c=3]{c} \textbf{Visualizza avvisi} \\
          \textbf{Goal in Context} & \SetCell[c=3]{l}{Un Dipedente vuole visualizzare un avviso}\\
            \textbf{Precodition} & \SetCell[c=3]{l}{Il Dipendente deve essere registrato e loggato nell'app}\\
          
            \textbf{Success End Condition} & \SetCell[c=3]{l}{Il Dipendente visualizza gli avvisi}\\
          
            \textbf{Failed End Condition}  & \SetCell[c=3]{l}{Il Dipendente non riesce a visualizzare i propri avvisi}\\
          
            \textbf{Primary Actor}  & \SetCell[c=3]{c}{Dipendente}\\
            \textbf{Trigger}  & \SetCell[c=3]{c}{Preme il pulsante  \emph{ "Avvisi "} nella propria Dashboard}\\
            
            \SetCell[r=3]{c}\textbf{Description}  & Step & UserAction & System\\
                                          & 1 & {Il Dipendente preme sul pulsante  \emph{ "Avvisi "}\\ sulla schermata \textbf{ \emph{M0boh_waiter}} o \textbf{ \emph{M0boh_sup}} \\ o \textbf{ \emph{M0boh_cucina}}} & \\
                                          & 2 &  & {Mostra la schermata \textbf{ \emph{M0boh_2}}}\\
          
            \SetCell[r=2]{c}{\textbf{Extension \#\ 1}\\ Non ci sono avvisi da mostrare}  & Step & UserAction & System\\
                                                         & 2a  &  & {Mostra la schermata \textbf{ \emph{M0boh_2err}}}\\

          
            \SetCell[r=3]{c}{\textbf{Subvariation \#\ 1}\\ Il Dipendente vuole marcare un avviso come visualizzato}  & Step & UserAction & System\\
                                                  & 1 & {Il Dipendente scorre col dito sull'avviso}  &  \\
                                                  & 2 &  & {Aggiorna la schermata \textbf{ \emph{M0boh_2}} eliminando l'avviso}\\

            %\textbf{Notes}  & \SetCell[c=3]{c}{\textcolor{red}{TODO}}\\
          \end{longtblr}
      \end{center}

