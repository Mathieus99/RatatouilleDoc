\subsection{Glossario}
    \begin{flushleft}
        In questa sezione vengono chiariti alcuni termini usati all'interno della documentazione, per rendere la lettura accessibile
        anche ai non esperti del settore.
    \end{flushleft}

    \begin{itemize}
        \item \emph{RecyclerView:} potente approccio nella risoluzione di un problema comune: la creazione di liste per la visualizzazione (su Android) di dati ottenuti da un servizio remoto o da un database locale.
        \item \emph{Adapter:} Un oggetto di tipo adapter in Android rappresenta un ponte tra un' AdapterView e i dati che questa deve rappresentare.
        \item \emph{Dialog:} Tipo di popup che il sistema Android mette a disposizione che mostra una finestra di dialog personalizzabile.
        \item \emph{Spring Boot:} Spring è uno strumento che permette a noi programmatori di usare il linguaggio di programmazione ad oggetti java per scrivere ottime app lato server.
        \item \emph{JPA:} Le Java Persistence API, talvolta riferite come JPA, sono un framework per il linguaggio di programmazione Java che si occupa della gestione della persistenza dei dati di un DBMS relazionale
        \item \emph{MVC:} Pattern Model-View-Controller.
        \item \emph{Three-Tier Architecture:} l'espressione architettura three-tier ("a tre strati") indica una particolare architettura software e hardware di tipo multi-tier per l'esecuzione di un'applicazione web che prevede la suddivisione dell'applicazione in tre diversi 
        moduli o strati dedicati rispettivamente alla interfaccia utente, alla logica funzionale (business logic) e alla gestione dei dati persistenti. Android e Spring-Boot ne supportano i principi.
        \item \emph{HTTP:} Protocollo di comunicazione client-server
        \item \emph{OpenFoodFacts:} Database online che raccoglie le informazioni(allergeni, ingredienti, ecc.) di molti cibi preconfezionati.
    \end{itemize}