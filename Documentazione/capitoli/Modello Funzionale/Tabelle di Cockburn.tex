\subsection{Modellazione dei casi d'uso} %sezione 2.7
    \begin{flushleft}
        Vengono qui riportate le tabelle di cockburn 
    \end{flushleft}


    \begin{table}[H]    
    \def\arraystretch{1.5}
    \begin{tabularx}{\linewidth}{|l|X|X|X|}
    
      \hline \textbf{Use Case \#1} & \multicolumn{3} {l|}
            {Aggiunta ristorante} \\
            
      \hline \textbf{Goal in Context} & \multicolumn{3}{>{\hsize=\dimexpr 3\hsize+4\tabcolsep+2\arrayrulewidth\relax}X|}
            { Un admin (proprietario di uno o più ristoranti) vuole aggiungere uno di questi nell' app Ratatuille.} \\
            
     \hline \textbf{Preconditions} & \multicolumn{3}{l|}
            {Il proprietario deve essere registrato e loggato nell'app come amministratore} \\
            
     \hline \textbf{Success End Conditions} & \multicolumn{3}{l|}
            {Il proprietario aggiunge correttamente il ristorante} \\
            
     \hline \textbf{Failed End Conditions} & \multicolumn{3}{l|}
            {Il proprietario non riesce ad aggiungere il ristorante} \\
            
     \hline \textbf{Primary Actor} & \multicolumn{3}{l|}
            {Amministratore} \\
            
     \hline \textbf{Trigger} & \multicolumn{3}{l|}
            {Preme il pulsante "Aggiungi"} \\
    
      \hline \multirow{2}{*}{\textbf{Description}} & \textbf{Step} & \textbf{User Action} & \textbf{System} \\
          \cline{2-4} &{1} & {L'amministratore preme sul tasto "AGGIUNGI RISTORANTE" sulla schermata \textbf{\textit{M04}}} & \\
          
          \cline{2-4} &{2} & {} & {Mostra la schermata \textbf{\textit{M06}}} \\
          
          \cline{2-4} &{3} & {Compila i campi necessari per la registrazione del proprio ristorante e preme sul tasto "SALVA" per aggiungere il Ristorante} & \\
          
          \cline{2-4} &{4} & & {Ricarica la schermata \textbf{\textit{M04}} aggiungendo nella lista dei ristoranti l'ultimo appena inserito} \\
    
      \hline


    
    \end{tabularx}
    
    \end{table}
    
        \begin{table}[H]    
    
        \def\arraystretch{1.5}
        
        
        \begin{tabularx}{\linewidth}{|X|X|X|X|}
        


         \hline \multirow{2}{*}{\textbf{Extension \#1}} & \textbf{Step} &
         \textbf{User Action} & \textbf{System} \\
         \cline{2-4} {L'Amministratore non fa nulla e torna indietro} &{4.a} & & {Torna alla schermata \textbf{\textit{M04}}} \\


         
          \hline\multirow{2}{*}{\textbf{Extension \#2}} & \textbf{Step} & \textbf{User Action} & \textbf{System} \\
         \cline{2-4} {L'amministratore ha aggiunto un Ristorante già presente nella propria lista di Ristoranti} & {4.b} & & {Mostra uno dei messaggi di errore della schermata \textbf{\textit{{M0boh:}}} restando allo step \textbf{2} dello scenario principale} \\
        

        \hline\multirow{2}{*}{\textbf{Extension \#3}} & \textbf{Step} & \textbf{User Action} & \textbf{System} \\
         \cline{2-4} {L'amministratore ha lasciato uno o più campi vuoti} & {4.c} & & {Mostra uno o più dei messaggi di errore della schermata \textbf{\textit{{M0boh}}} restando allo step \textbf{2} dello scenario principale} \\

         
        \hline\multirow{2}{*}{\textbf{Extension \#4}} & \textbf{Step} & \textbf{User Action} & \textbf{System} \\
         \cline{2-4} {L'amministratore ha aggiunto un nome troppo corto} & {4.d} & & {Mostra il messaggio di errore di fianco al campo "Nome" della schermata \textbf{\textit{{M0boh}}} restando allo step \textbf{3} dello scenario principale} \\
         

         \hline\multirow{2}{*}{\textbf{Extension \#5}} & \textbf{Step} & \textbf{User Action} & \textbf{System} \\
         \cline{2-4} {L'amministratore ha aggiunto un numero di coperti non valido} & {4.e} & & {Mostra il messaggio di errore di fianco al campo "Numero di coperti" della schermata \textbf{\textit{{M0boh}}} restando allo step \textbf{3} dello scenario principale} \\
         

         \hline\multirow{2}{*}{\textbf{Extension \#6}} & \textbf{Step} & \textbf{User Action} & \textbf{System} \\
         \cline{2-4} {L'amministratore ha aggiunto un indirizzo troppo corto} & {4.f} & & {Mostra il messaggio di errore di fianco al campo "Indirizzo" della schermata \textbf{\textit{{M0boh}}} restando allo step \textbf{3} dello scenario principale} \\
         
         \hline
        
        \end{tabularx}
        
        \end{table}
        
        \begin{table}[H]    
    
        \def\arraystretch{1.5}
        
        
        \begin{tabularx}{\linewidth}{|X|X|X|X|}

         
        \hline\multirow{2}{*}{\textbf{Extension \#7}} & \textbf{Step} & \textbf{User Action} & \textbf{System} \\
         \cline{2-4} {L'amministratore ha aggiunto un numero di telefono non valido} & {4.g} & & {Mostra il messaggio di errore di fianco al campo "Numero di telefono" della schermata \textbf{\textit{{M0boh}}} restando allo step \textbf{3} dello scenario principale} \\
         

         \hline \multirow{2}{*}{\textbf{Subvariation \#1}} & \textbf{Step} & \textbf{User Action} & \textbf{System} \\
         \cline{2-4} {L' amministratore torna indietro completando solo parzialmente i campi} & {1} & & {Mostra la schermata \textbf{\textit{M0boh}}} \\
         \cline{2-4} & {2} & {Preme su "SI"} &  \\
         \cline{2-4} & {3} & & {Torna alla schermata \textbf{\textit{M04}}} \\

         
        \hline \multirow{2}{*}{\textbf{Subvariation \#2}} & \textbf{Step} & \textbf{User Action} & \textbf{System} \\
         \cline{2-4} {L' amministratore inizialmente vuole tornare indietro completando solo parzialmente i campi ma cambia idea} & {1} & & {Mostra la schermata \textbf{\textit{M0boh}}} \\
         \cline{2-4} & {2} & {Preme su "NO"} &  \\
         \cline{2-4} & {3} & & {Torna alla schermata \textbf{\textit{M06}} allo step \textbf{2} dello scenario principale} \\
        \hline       
        
        \end{tabularx}
        
        \end{table} 

        \newpage
        
        \begin{center}
              \begin{tblr}{hlines = {0.9pt}, vlines = {0.9pt}, colspec = {X[c]X[c]X[c]X[c]}, width = \textwidth}
                     \textbf{Use Case \#2}       & \SetCell[c=3]{c} \textbf{Aggiunta piatto al menù} \\

                     \textbf{Goal in Context}    & \SetCell[c=3]{l}{Un admin (proprietario di uno o più ristoranti) vuole aggiungere un piatto \\in un menu di un suo ristorante}\\
                     
                     \textbf{Precodition}        & \SetCell[c=3]{l}{Il proprietario deve essere registrato e loggato nell'app come amministratore, \\deve aver aggiunto almeno un ristorane e deve averne creato un menu}\\
                     
                     \textbf{Success End Condition} & \SetCell[c=3]{l}{Il proprietario aggiunte correttamente un piatto al suo menu}\\
                     
                     \textbf{Failed End Condition}  & \SetCell[c=3]{l}{Il proprietario non riesce ad aggiungere un piatto al suo ristorante}\\
                     
                     \textbf{Primary Actor}  & \SetCell[c=3]{c}{Amministratore}\\
                     \textbf{Trigger}  & \SetCell[c=3]{c}{Preme il pulsante "Accetta"}\\
                     
                     \SetCell[r=4]{c}\textbf{Description}  & Step & UserAction & System\\
                                                        & 1 & {L'amministratore preme sul tasto \textit{"AGGIUNGI RISTORANTE"}\\ sulla schermata \textbf{\textit{M04}}} & \\
                                                        & 2 &  TODO &     \\
                                                        & 3 &       & TODO\\
                     
                     \SetCell[r=2]{c}\textbf{Extension} & Step & UserAction & System\\
                                                        & Step & UserAction & System\\
                     
                     \SetCell[r=2]{c}\textbf{Subvariation}  & Step & UserAction & System\\
                                                               & Step & UserAction & System\\
                     
                     \textbf{Notes}  & \SetCell[c=3]{c}{\textcolor{red}{TODO}}\\
              \end{tblr}
        \end{center}