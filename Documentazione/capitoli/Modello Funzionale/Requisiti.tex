\section{Modello Funzionale}
    \subsection{Requisiti Funzionali}
        \begin{flushleft}  
            {\large
                Vengono qui presentati i requisiti funzionali dell'applicativo, ossia quei servizi che l'app  deve offrire agli utenti:
            } 
        \end{flushleft}
        
        \subsubsection{Admin}
        

        \begin{center}
          % Admin 1
          \begin{tblr}{hlines = {0.9pt}, vlines = {0.9pt}, row{1} = {marroneApp!60}, colspec = {X[c]X[l]}, width = \textwidth}
                  \textbf{ID}         & Admin\_1                             \\
                  \textbf{Nome}       & Registrazione account amministratore \\
                  \textbf{Descrizione} & {Il sistema permette ad un amministratore non registrato di registrarsi alla\\ piattaforma utilizzando: \textit{nome}, \textit{cognome}, \textit{email}, \textit{codice fiscale}, \textit{P.IVA} e \textit{password}}
          \end{tblr}

          \vspace{1cm}

          % Admin 2
          \begin{tblr}{hlines = {0.9pt}, vlines = {0.9pt}, row{1} = {marroneApp!60}, colspec = {X[c]X[l]}, width = \textwidth}
                  \textbf{ID}         & Admin\_2                             \\
                  \textbf{Nome}       & Modifica account amministratore \\
                  \textbf{Descrizione} & {Il sistema permette ad un amministratore loggato di modifcare i campi del proprio account}
          \end{tblr}

          \vspace{1cm}

          % Admin 3
          \begin{tblr}{hlines = {0.9pt}, vlines = {0.9pt}, row{1} = {marroneApp!60}, colspec = {X[c]X[l]}, width = \textwidth}
                  \textbf{ID}         & Admin\_3                             \\
                  \textbf{Nome}       & Registrazione dei dipendenti \\
                  \textbf{Descrizione} & {Il sistema permette ad un amministratore di creare utenze per i dipendenti non registrati del ristorante,\\ specificandone \textit{nome}, \textit{cognome}, \textit{email} e \textit{ruolo}}
          \end{tblr}

          \vspace{1cm}

          % Admin 4
          \begin{tblr}{hlines = {0.9pt}, vlines = {0.9pt}, row{1} = {marroneApp!60}, colspec = {X[c]X[l]}, width = \textwidth}
                  \textbf{ID}          & Admin\_4                             \\
                  \textbf{Nome}        & Modificare/Eliminare account dipendenti  \\
                  \textbf{Descrizione} & {Il sistema permette ad un amministratore loggato di modificare gli account dei propri dipendenti.}
          \end{tblr}

          \vspace{1cm}

          % Admin 5
          \begin{tblr}{hlines = {0.9pt}, vlines = {0.9pt}, row{1} = {marroneApp!60}, colspec = {X[c]X[l]}, width = \textwidth}
                  \textbf{ID}          & Admin\_5                             \\
                  \textbf{Nome}        & Aggiungere personale della cucina  \\
                  \textbf{Descrizione} & {Il sistema permette ad un amministratore loggato di aggiungere al proprio ristorante il personale della cucina.}
          \end{tblr}

          \vspace{1cm}

          % Admin 6
          \begin{tblr}{hlines = {0.9pt}, vlines = {0.9pt}, row{1} = {marroneApp!60}, colspec = {X[c]X[l]}, width = \textwidth}
                  \textbf{ID}          & Admin\_6                             \\
                  \textbf{Nome}        & Aggiunta dei Ristoranti  \\
                  \textbf{Descrizione} & {Il sistema permette ad un amministratore di poter aggiungere le proprie attività di ristorazione (CAMPI DA DEFINIRE).}
          \end{tblr}

          \vspace{1cm}

          % Admin 7
          \begin{tblr}{hlines = {0.9pt}, vlines = {0.9pt}, row{1} = {marroneApp!60}, colspec = {X[c]X[l]}, width = \textwidth}
                  \textbf{ID}          & Admin\_7                             \\
                  \textbf{Nome}        &  Modifica/Eliminazione dei Ristoranti  \\
                  \textbf{Descrizione} & {Il sistema permette ad un amministratore di poter modificare ed eliminare le proprie attività di ristorazione del sistema.}
          \end{tblr}

          \vspace{1cm}

          % Admin 8
          \begin{tblr}{hlines = {0.9pt}, vlines = {0.9pt}, row{1} = {marroneApp!60}, colspec = {X[c]X[l]}, width = \textwidth}
                  \textbf{ID}          & Admin\_8                             \\
                  \textbf{Nome}        &  Modifica dati dei Dipendenti  \\
                  \textbf{Descrizione} & {Il sistema permette ad un amministratore loggato di poter cambiare i dati personali dei dipendenti (nome, cognome, email, luogo).}
          \end{tblr}

          \vspace{1cm}

          % Admin 9
          \begin{tblr}{hlines = {0.9pt}, vlines = {0.9pt}, row{1} = {marroneApp!60}, colspec = {X[c]X[l]}, width = \textwidth}
                  \textbf{ID}          & Admin\_9                             \\
                  \textbf{Nome}        &  Aggiungere/Modificare elementi nel menù  \\
                  \textbf{Descrizione} & {Il sistema permette ad un amministratore o supervisore dell'attività di ristorazione di aggiungere/modificare elementi nel menù dell'attività. Ogni elemento dovrà avere i seguenti campi:\\ Nome, Costo, Descrizione, Elenco di Allergeni, Categoria/e}
          \end{tblr}

          \vspace{1cm}

          % Admin 10
          \begin{tblr}{hlines = {0.9pt}, vlines = {0.9pt}, row{1} = {marroneApp!60}, colspec = {X[c]X[l]}, width = \textwidth}
                  \textbf{ID}          & Admin\_10                             \\
                  \textbf{Nome}        &  Modifica dati personali\\
                  \textbf{Descrizione} & {Il sistema permette ad un amministratore loggato di poter cambiare i propri dati  personali.}
          \end{tblr}

          \vspace{1cm}

          % Admin 12
          \begin{tblr}{hlines = {0.9pt}, vlines = {0.9pt}, row{1} = {marroneApp!60}, colspec = {X[c]X[l]}, width = \textwidth}
                  \textbf{ID}          & Admin\_12                             \\
                  \textbf{Nome}        &  Tradurre il menu           \\
                  \textbf{Descrizione} & {Il sistema permette ad un Amministratore di poter tradurre gli elementi del proprio menù in un altra lingua}
          \end{tblr}

          \vspace{1cm}

          % Admin 13
          \begin{tblr}{hlines = {0.9pt}, vlines = {0.9pt}, row{1} = {marroneApp!60}, colspec = {X[c]X[l]}, width = \textwidth}
                  \textbf{ID}          & Admin\_13                             \\
                  \textbf{Nome}        &  Visualizza statistiche personale della cucina\\
                  \textbf{Descrizione} & {Il sistema permette ad un Amministratore di visualizzare, grazie anche all'ausilio di grafici interattivi, informazioni sull'operato degli addetti alla cucina}
          \end{tblr}

          \vspace{1cm}

          % Admin 14
          \begin{tblr}{hlines = {0.9pt}, vlines = {0.9pt}, row{1} = {marroneApp!60}, colspec = {X[c]X[l]}, width = \textwidth}
                  \textbf{ID}          & Admin\_14                             \\
                  \textbf{Nome}        & Modifica menu \\
                  \textbf{Descrizione} & {Il sistema permette ad un Amministratore di poter aggiornare (aggiungere, modificare ed eliminare elementi) il menu del ristorante}
          \end{tblr}
        \end{center}

        \subsubsection{Cameriere}
        \begin{center}

          % Waiter 1
          \begin{tblr}{hlines = {0.9pt}, vlines = {0.9pt}, row{1} = {marroneApp!60}, colspec = {X[c]X[l]}, width = \textwidth}
                  \textbf{ID}          & Waiter\_1                             \\
                  \textbf{Nome}        & Prendere le ordinazioni \\
                  \textbf{Descrizione} & {Il sistema permette ai camerieri di prendere ordinazioni ai tavoli, inoltrandole alla cucina.}
          \end{tblr}

          \vspace{1cm}

          % Waiter 2
          \begin{tblr}{hlines = {0.9pt}, vlines = {0.9pt}, row{1} = {marroneApp!60}, colspec = {X[c]X[l]}, width = \textwidth}
                  \textbf{ID}          & Waiter\_2                             \\
                  \textbf{Nome}        & Gestione delle ordinazioni \\
                  \textbf{Descrizione} & {Il sistema permette ai camerieri di prendere ordinazioni ai tavoli, inoltrandole alla cucina.}
          \end{tblr}

          \vspace{1cm}

          % Waiter 3
          \begin{tblr}{hlines = {0.9pt}, vlines = {0.9pt}, row{1} = {marroneApp!60}, colspec = {X[c]X[l]}, width = \textwidth}
                  \textbf{ID}          & Waiter\_3                             \\
                  \textbf{Nome}        & Evasione degli ordini \\
                  \textbf{Descrizione} & {Il sistema permette a un Cameriere di marcare i singoli elementi di un ordine come conclusi, aggiornando gli addetti in cucina}
          \end{tblr}

          \vspace{1cm}

          % Waiter 4
          \begin{tblr}{hlines = {0.9pt}, vlines = {0.9pt}, row{1} = {marroneApp!60}, colspec = {X[c]X[l]}, width = \textwidth}
                  \textbf{ID}          & Waiter\_4                             \\
                  \textbf{Nome}        & Sollecitare la cucina \\
                  \textbf{Descrizione} & {Il sistema permette a un Cameriere di sollecitare la cucina nel caso in cui un ordine sia da troppo tempo in preparazione }
          \end{tblr}

        \end{center}
        
        \subsubsection{Cucina}

        \begin{center}
          % Kitchen 1
          \begin{tblr}{hlines = {0.9pt}, vlines = {0.9pt}, row{1} = {marroneApp!60}, colspec = {X[c]X[l]}, width = \textwidth}
                  \textbf{ID}          & Kitchen\_1                             \\
                  \textbf{Nome}        & Marcare gli ordini pronti \\
                  \textbf{Descrizione} & {Il sistema permette alla cucina di marcare gli ordini pronti alla "consegna", specificando lo chef che l'ha preparato}
          \end{tblr}

          \vspace{1cm}

          % Kitchen 2
          \begin{tblr}{hlines = {0.9pt}, vlines = {0.9pt}, row{1} = {marroneApp!60}, colspec = {X[c]X[l]}, width = \textwidth}
                  \textbf{ID}          & Kitchen\_2                             \\
                  \textbf{Nome}        & Sollecitare i camerieri \\
                  \textbf{Descrizione} & {Il sistema permette alla cucina di notificare i camerieri nel caso in cui un ordine sia pronto alla consegna da troppo tempo }
          \end{tblr}

        \end{center}

        \subsubsection{Supervisore}

        \begin{center}

          % Hypervisor 1 (Forse Supervisor (?))
          \begin{tblr}{hlines = {0.9pt}, vlines = {0.9pt}, row{1} = {marroneApp!60}, colspec = {X[c]X[l]}, width = \textwidth}
                  \textbf{ID}          & Hypervisor\_1                             \\
                  \textbf{Nome}        & Avvisare il personale \\
                  \textbf{Descrizione} & {Il sistema permette ad un supervisore ed un amministratore di inviare degli avvisi al personale}
          \end{tblr}

          \vspace{1cm}

          % Hypervisor 2 (Forse Supervisor (?))
          \begin{tblr}{hlines = {0.9pt}, vlines = {0.9pt}, row{1} = {marroneApp!60}, colspec = {X[c]X[l]}, width = \textwidth}
                  \textbf{ID}          & Hypervisor\_2                             \\
                  \textbf{Nome}        & Visualizzare stato ordini per tavolo\\
                  \textbf{Descrizione} & {Il sistema permette ad un Supervisore ed un Cameriere di visualizzare gli stati delle ordinazioni per ogni tavolo}
          \end{tblr}

          \vspace{1cm}

          % Hypervisor 3 (Forse Supervisor (?))
          \begin{tblr}{hlines = {0.9pt}, vlines = {0.9pt}, row{1} = {marroneApp!60}, colspec = {X[c]X[l]}, width = \textwidth}
                  \textbf{ID}          & Hypervisor\_3                             \\
                  \textbf{Nome}        & Visualizzare ordini in arrivo\\
                  \textbf{Descrizione} & {Il sistema permette ad un Supervisore e alla Cucina di visualizzare l'elenco di ordini in arrivo}
          \end{tblr}

          \vspace{1cm}

          % Hypervisor 4 (Forse Supervisor (?))
          \begin{tblr}{hlines = {0.9pt}, vlines = {0.9pt}, row{1} = {marroneApp!60}, colspec = {X[c]X[l]}, width = \textwidth}
                  \textbf{ID}          & Hypervisor\_4                             \\
                  \textbf{Nome}        & Visualizzare ordini in uscita\\
                  \textbf{Descrizione} & {Il sistema permette ad un Supervisore e alla Cucina di visualizzare l'elenco di ordini pronti all'uscita}
          \end{tblr}

          \vspace{1cm}

          % Hypervisor 5 (Forse Supervisor (?))
          \begin{tblr}{hlines = {0.9pt}, vlines = {0.9pt}, row{1} = {marroneApp!60}, colspec = {X[c]X[l]}, width = \textwidth}
                  \textbf{ID}          & Hypervisor\_5                             \\
                  \textbf{Nome}        & Visualizzare storico ordini\\
                  \textbf{Descrizione} & {Il sistema permette ad un Supervisore e alla Cucina di visualizzare lo storico degli ordini}
          \end{tblr}

          \vspace{1cm}
        \end{center}


        \subsubsection{Tutti}

        \begin{center}

          % All 1
          \begin{tblr}{hlines = {0.9pt}, vlines = {0.9pt}, row{1} = {marroneApp!60}, colspec = {X[c]X[l]}, width = \textwidth}
                  \textbf{ID}          & All\_1                             \\
                  \textbf{Nome}        & Recupero/Cambio Password\\
                  \textbf{Descrizione} & {Il sistema permette a tutti gli utenti registrati di poter recuperare la password e di poterla modificare}
          \end{tblr}

          \vspace{1cm}

          % All 2
          \begin{tblr}{hlines = {0.9pt}, vlines = {0.9pt}, row{1} = {marroneApp!60}, colspec = {X[c]X[l]}, width = \textwidth}
                  \textbf{ID}          & All\_2                             \\
                  \textbf{Nome}        & Login\\
                  \textbf{Descrizione} & {Il sistema permette a tutti gli utenti registrati di poter effettuare il login con Username e Password all'interno della piattaforma}
          \end{tblr}

          \vspace{1cm}

          % All 3
          \begin{tblr}{hlines = {0.9pt}, vlines = {0.9pt}, row{1} = {marroneApp!60}, colspec = {X[c]X[l]}, width = \textwidth}
                  \textbf{ID}          & All\_3                             \\
                  \textbf{Nome}        & Visualizzare un avviso o sollecitazione\\
                  \textbf{Descrizione} & {Il sistema permette a tutti gli utenti registrati pi poter visualizzare gli avvisi e le sollecitazioni ricevuti, marcandoli come visualizzati}
          \end{tblr}
        \end{center}

        \newpage

        \subsection{Requisiti Non Funzionali}
        \begin{flushleft} Vengono qui elencati i requisiti non funzionali dell'applicativo: \end{flushleft}

        \begin{center}
          % Unfunctional 1
          \begin{tblr}{hlines = {0.9pt}, vlines = {0.9pt}, row{1} = {marroneApp!60}, colspec = {X[c]X[l]}, width = \textwidth}
                  \textbf{ID}          & Unfunctional\_1                             \\
                  \textbf{Nome}        & Policy first password\\
                  \textbf{Descrizione} & {Il sistema richiede al primo accesso di un dipendente il cambio della password provvisoria in una password personale.}
          \end{tblr}

          \vspace{1cm}

          % Unfunctional 2
          \begin{tblr}{hlines = {0.9pt}, vlines = {0.9pt}, row{1} = {marroneApp!60}, colspec = {X[c]X[l]}, width = \textwidth}
                  \textbf{ID}          & Unfunctional\_2                             \\
                  \textbf{Nome}        & Verifica esistenza della mail\\
                  \textbf{Descrizione} & {Al fine di evitare registrazioni con e-mail fittizie, il sistema richiede l'autenticazione della mail mediante codice di verifica per poter procedere con il completamento della registrazione}
          \end{tblr}

          \vspace{1cm}

          % Unfunctional 3
          \begin{tblr}{hlines = {0.9pt}, vlines = {0.9pt}, row{1} = {marroneApp!60}, colspec = {X[c]X[l]}, width = \textwidth}
                  \textbf{ID}          & Unfunctional\_3                             \\
                  \textbf{Nome}        & Password Strength\\
                  \textbf{Descrizione} & {Al fine di evitare la creazione di password poco sicure, il sistema impone all'utente di utilizzare una password di almeno 8 caratteri che contenga numeri e caratteri speciali.}
          \end{tblr}

          \vspace{1cm}

          % Unfunctional 4
          \begin{tblr}{hlines = {0.9pt}, vlines = {0.9pt}, row{1} = {marroneApp!60}, colspec = {X[c]X[l]}, width = \textwidth}
                  \textbf{ID}          & Unfunctional\_4                             \\
                  \textbf{Nome}        & \textcolor{red}{TODO}\\
                  \textbf{Descrizione} & {Una P.IVA appartiene ad un solo amministratore.}
          \end{tblr}
        \end{center}

        \subsection{Requisiti di Dominio}

        \begin{center}
          % Domain 1
          \begin{tblr}{hlines = {0.9pt}, vlines = {0.9pt}, row{1} = {marroneApp!60}, colspec = {X[c]X[l]}, width = \textwidth}
                  \textbf{ID}          & Domain\_1                             \\
                  \textbf{Nome}        & GDPR\\
                  \textbf{Descrizione} & {Il sistema deve essere conforme alla normativa GDPR (Regolamento Generale  sulla Protezione dei Dati), per il trattamento dei dati personali e riguardante la privacy dell'utente}
          \end{tblr}

          \vspace{1cm}

          % Domain 2
          \begin{tblr}{hlines = {0.9pt}, vlines = {0.9pt}, row{1} = {marroneApp!60}, colspec = {X[c]X[l]}, width = \textwidth}
                  \textbf{ID}          & Domain\_2                             \\
                  \textbf{Nome}        & NORME ISO\\
                  \textbf{Descrizione} & {\textcolor{red}{TODO}}
          \end{tblr}
        \end{center}
